\font\bigtenrm=cmr10 scaled\magstep1
\font\bigsl=cmsl10 scaled\magstep1
\font\bigit=cmti10 scaled\magstep1
\font\bigbf=cmbx10 scaled\magstep1
\font\type=cmtt10 scaled\magstep1
\hsize 16.1cm \vsize 22cm
\baselineskip 18pt
\leftskip 0.5in
\bigskip
\bigbf
\bigskip

\centerline{\hfil Minimal Basis Sets:  Practical LMTO Downfolding\hfil}
\bigskip
\bigtenrm
\centerline{\hfil A. T. Paxton and Mark van Schilfgaarde,
SRI International\hfil}
\vskip5pt
\centerline{\hfil 333 Ravenswood Ave, Menlo Park, CA 94025\hfil}
\vskip15pt
\centerline{\hfil O. K. Andersen, Max-Planck Institut f\"ur
Festk\"orperforschung\hfil}
\vskip5pt
\centerline{\hfil 7000 Stuttgart-80\hfil}
\bigskip

These notes describe the practical considerations involved in
incorporating orbital downfolding into an existing LMTO program.
The starting point is assumed to be a working code in the tight-binding
representation, here denoted $\alpha$. However, the development is
equally valid for any starting representation, so that, for example,
one could set $\alpha=0$ in all that follows.

Downfolding is effected by a basis transformation to an LMTO
representation we will call $\beta$, defined as
$$\beta_l = \alpha_l;\qquad\beta_i=1/P^0(e_\nu)$$
where $l$, $i$ refer to the lower and intermediate
sets respectively and $\alpha$ are the conventional tight-binding
screening constants.
The $i$-orbitals are those which will be downfolded
so that they do not contribute to the dimension of the secular problem.
$1/P^0(e_\nu)$ is the inverse potential function (which
may be thought of as minus the tangent of the KKR phase shift) evaluated
at $e_\nu$.

Lambrecht and Andersen have shown that orbital downfolding amounts
to a linearisation of $1/P^0$ about the energy $V^\sigma$
for any general representation $\sigma$, including $\sigma=0$.
$V^\sigma$ is the energy at which the radial solution $\varphi^\sigma$
has logarithmic derivative equal to $\ell$.

The choice of $\beta$-representation has the main advantage that $1/P^0$
is linearised about $V^\beta=e_\nu$ which leads to the smallest error in
the linear method.  Two other advantages are that the moments of the
charge density in the $l$ and $i$ sets are uncoupled; and that the
$i$-wave eigenvectors, three-centre integrals and $\dot\beta_i$ can be
expressed in a representation-independent form (these will be discussed
below).

In a self-consistent LMTO-ASA procedure, the downfolded band
calculation comprises the following steps. 1. Transformation of
the structure constants to the $\beta$-representation. (This
depends on the potential from the previous iteration.) 2. Assembling
the overlap and hamiltonian matrices. 3. Computing eigenvalues and
eigenvectors. 4. Accumulating the moments of the new charge density.

\vskip 10pt
\line {1. Transforming the structure constants\hfil}
\vskip 5pt

We begin with the Bloch transformed, tight-binding structure constants,
$S^\alpha$, and their energy derivatives at
$\kappa^2=0$, $\dot S^\alpha$.
We need not invert the whole structure constant matrix as might appear
necessary from the usual expressions for the change of basis, because
only the screening constants in the $i$-set are different from the
tight-binding values. In this case, Andersen has shown
(Varenna notes, p.~103) that if we
partition the structure constant matrices into square
$l$-wave and $i$-wave blocks (denoted by subscripts $ii$ and $ll$)
connected by a rectangular $il$ block, then
defining the vector $\xi_i=(\beta_i-\alpha_i)$ and
$A_{il}={(\xi^{-1}_i - S^\alpha_{ii})}^{-1} S^\alpha_{il}$, one has
$$S^\beta_{ll}=S^\alpha_{ll} + (S^\alpha_{il})^\dagger A_{il}$$
and
$$S^\beta_{il}=\xi^{-1}_i A_{il}$$
Note that in the present case, $\xi^{-1}_i$ is the vector of
inverse potential functions at $e_\nu$, of the $i$-waves, in the
tight-binding representation.

In order to transform ${\dot S}^\alpha$ we need the vector
$\dot\xi_i=(\dot\beta_i-\dot\alpha_i)$,
where the choice of $\dot\xi_i$ is
arbitrary. We choose $\dot\xi_i$ so as to cause three-centre overlap
integrals over the $i$-waves to vanish as will be discussed below.
If $\dot\beta_l=\dot\alpha_l$, Andersen has shown that
$${\dot S}^\beta_{ll} = {\dot S}^\alpha_{ll}
                               + A_{il}^\dagger {\dot S}^\alpha_{il}
                               + ({\dot S}^\alpha_{il})^\dagger A_{il}
-A_{il}^\dagger ({\dot\xi}^{-1}_i - {\dot S}^\alpha_{ii}) A_{il}$$
and
$${\dot S}^\beta_{il} = -({\dot\xi}^{-1}_i - {\dot S}^\alpha_{ii}) A_{il}$$
where ${\dot\xi}^{-1}_i=-\xi^{-1}_i {\dot\xi}_i \xi^{-1}_i$.

\vskip 10pt
\line {2. Overlap and Hamiltonian\hfil}
\nobreak\vskip 5pt\nobreak
It is convenient to divide $O$ and $H$ into ASA and CC (Combined
Correction) contributions.  Each of these has three terms to third order
{\it viz.,} one- two- and three-centre integrals.  Here we give explicit
expressions for each of the six terms in the downfolded matrices.  Due
to our choice of $\beta$ and $\dot\beta$, we may express all quantities
in terms of $S^\beta$ and ${\dot S}^\beta$, $e_\nu$, and the five
traditional tight-binding parameters, $d^\alpha$, $c^\alpha$, $o^\alpha$
and $p^\alpha$ (see Varenna notes).  The reason is that it turns out
that in expressions for $i$-wave three-centre integrals, eigenvectors
and $\dot\xi_i$, the potential parameters
appear only in the combination,
$${{\sqrt\Delta}\over{C-e_\nu}}=
{\sqrt{d^\alpha}\over{c^\alpha-e_\nu}}$$
which is representation-independent,
and is equal to the potential parameter
$\Gamma^{-\scriptstyle{1\over 2}}$
when in the representation $\beta_i$ (that is, when $e_\nu=V^\beta$).
Thus we have, for the $i$-waves only,
$${\sqrt{d^\alpha}\over{c^\alpha-e_\nu}}=
{1\over\sqrt{\Gamma^\beta}}$$
Therefore we may conveniently use the tight-binding potential
parameters even for the $i$-waves, and in
what follows we supress the superscript $\alpha$ on these.
(This is consistent with the present development being independent
of the original representation.) It is convenient to
define matrices,
$$S_d\equiv\sqrt{d}S^\beta\sqrt{d}\qquad{\rm and}\qquad
\dot S_d\equiv\sqrt{d}\dot S^\beta\sqrt{d}$$
and for the
CC-terms, we write the diagonal matrices that appear in eq.~3.87 in the
Kanpur notes as follows.
$$D^{(0)}={2\over{2\ell -1}}\biggl({w\over s}\biggr)^{2\ell -1}$$
$$D^{(1)}={{(s/w)^2}\over{2(2\ell +1)}} + \beta{2\over{2\ell -1}}
  \biggl({w\over s}\biggr)^{2\ell -1}$$
$$D^{(2)}=-{1\over{2(2\ell +1)^2(2\ell +3)}}
  \biggl( {s\over w}\biggr)^{2\ell +3}
  + \beta{(s/w)^2\over {2\ell +1}}
  + \beta^2{2\over{2\ell -1}}\biggl({w\over s}\biggr)^{2\ell -1}
  + w^{-2}\dot\alpha$$
where $s$ and $w$ are atomic sphere radius and average Wigner-Seitz
radius.  Note, they are constructed with $\dot\alpha$ for both the $l$-
and $i$-sets (this is simply for computational convenience).  These can
be made once the screening constants in the $i$-channels have been
replaced by the inverse potential functions.

The orbitals that are folded down can only provide $\dot\varphi$-like
tails to the basis, so that in these channels there is only an
$\Omega$ matrix and no $\Pi$ matrix (see eq. 2.25a -- Ghent notes and
eq. 2.26 -- Kanpur notes). Therefore once the $i$-waves have been folded
in to the structure constants during the basis transformation, their
only explicit contributions to $O$ and $H$ come {\it via} the
three-centre integrals $\Omega^\dagger p \Omega$  and
$\Omega^\dagger e_\nu p \Omega$  and three-centre CC terms.
We now choose $\dot\xi_i$ so that these terms in $O$ sum to zero.
By setting the terms in brackets in eq. 3.90 -- Kanpur notes to zero
(with $V^\beta=e_\nu$) we arrive at the condition
$$\dot\xi_i =-w^2 D^{(2)}_i - {1\over\Gamma^\beta_i}$$
in which $\dot\xi_i$ depends only
on representation through the explicit
appearance of $\beta$ in $D^{(2)}$ and not through the potential
parameters.

Three-centre integrals over the $i$-set can also be cast in a
representation-independent form: using eq. 3.91 -- Kanpur notes, we
have for the $i$-wave hamiltonian entries,
$$\sqrt{d_l}(S^\beta_{il})^\dagger
\biggl({e_\nu-V_{\rm mtz}\over\Gamma^\beta_i}\biggr)
S^\beta_{il}\sqrt{d_l}$$
where $V_{\rm mtz}$ is the muffin-tin zero of energy.
Note that the CC term in $D^{(2)}_i$ has been cancelled by our choice of
$\dot\xi_i$ which has elliminated all $i$-wave three-centre terms in $O$
and all but the ASA three-centre terms in $H$.

The remaining contribution to $O$ and $H$ come from the $l$-waves, and
have the traditional form as follows.
\vskip 5pt
\line{Overlap ASA:\hfil}
\vskip2pt
\line{One-centre\hfil $1+2(c-e_\nu)o+p(c-e_\nu)^2$\hfil}
\vskip2pt
\line{Two-centre\hfil $S_d \bigl(o+p(c-e_\nu)\bigr)+
                       \bigl(o+p(c-e_\nu)\bigr)S_d$\hfil}
\vskip2pt
\line{Three-centre\hfil $S_d p S_d$\hfil}
\vskip 5pt
\line{Overlap CC:\hfil}
\vskip2pt
\line{One-centre\hfil $w^2 D^{(0)} d$\hfil}
\vskip2pt
\line{Two-centre\hfil $w^2 (S_d D^{(1)} + D^{(1)} S_d)$\hfil}
\vskip2pt
\line{Three-centre\hfil $w^2 S_d (D^{(2)}/d) S_d - \dot S_d$\hfil}
\vskip 5pt
\line{Hamiltonian ASA:\hfil}
\vskip2pt
\line{One-centre\hfil $c+(c-e_\nu)\bigl(2oe_\nu
                        +(c-e_\nu)(o+pe_\nu)\bigr)$\hfil}
\vskip2pt
\line{Two-centre\hfil $S_d
      \bigl({\scriptstyle 1\over 2}+oe_\nu+(c-e_\nu)(o+pe_\nu)\bigr)+
      \bigl({\scriptstyle 1\over 2}+oe_\nu+(c-e_\nu)(o+pe_\nu)\bigr)
                          S_d$\hfil}
\vskip2pt
\line{Three-centre\hfil $S_d(o+pe_\nu)S_d$\hfil}
\vskip 5pt
\line{Hamiltonian CC:\hfil}
\vskip2pt
\line{One-centre\hfil $V_{\rm mtz} w^2 D^{(0)} d$\hfil}
\vskip2pt
\line{Two-centre\hfil $V_{\rm mtz}w^2(S_d D^{(1)}+D^{(1)}S_d)$\hfil}
\vskip2pt
\line{Three-centre\hfil $V_{\rm mtz}\bigl(w^2 S_d(D^{(2)}/d)S_d
                         - \dot S_d\bigr)$\hfil}
\vskip5pt
Formulating the problem in this way has the advantage that downfolding
can be incorporated into a standard tight-binding LMTO program with a
minimal number of changes. The only difficult part is the scatter/gather
and index pointing needed in partitioning the $S$, $O$ and $H$ matrices and
the potential parameter, screening constant and CC vectors.
\vskip 10pt
\line{3. Eigenvalues and Eigenvectors\hfil}
\nobreak\vskip 5pt\nobreak
Although only the $ll$ block of the hamiltonian is diagonalised, one
may obtain an expression for the zero-order eigenvectors in the $i$-set
in terms of the eigenvectors of the $l$-waves. (See, for example,
Varrena notes, eq.~147.)
The success of orbital downfolding is due to the effect of the basis
transformation which is to make the tails of the $\dot\varphi$-like
$i$-waves mimic as closely as possible the actual $\varphi$ orbitals in
the $i$-set had they remained in the basis.  The way to achieve this is
to make $o^\beta$ as large as possible so that in the transformation
$\dot\varphi^\beta_i=\dot\varphi_i+\varphi_i o^\beta$ the last term
dominates.  In fact, in the $\beta_i$-representation in which
downfolding is most effective, the conventional potential parameters $d$
and $c-e_\nu$ are zero, whilst $o$ and $p$ are infinite. (This is why we
have emphasised the use of these parameters in the original
$\alpha$-representation.) The first-order hamiltonian, $h^\beta=c^\beta
-e_\nu + \sqrt{d^\beta}S^\beta\sqrt{d^\beta}$ is therefore zero in the
$i$-set, but $o^\beta h^\beta$ is finite and $\varphi^\beta_i$ can be
properly normalised if the LMTO
$$\vert\chi\rangle=\vert\varphi_l\rangle+\vert\dot\varphi_l\rangle
h_{ll}+\vert\dot\varphi_i\rangle h_{il}+\vert\kappa\rangle$$
becomes
$$\vert\chi^\beta\rangle=\vert\varphi_l\rangle+
\vert\dot\varphi_l^\beta\rangle h_{ll}^\beta
+\vert\varphi_i^\beta\rangle {-1\over\sqrt{\Gamma^\beta_i}}
S_{il}^\beta\sqrt{d_l}+\vert\kappa^\beta\rangle$$
where the expansion coefficients of the $\varphi^\beta_i$ are the
correct limit of $o^\beta_i h^\beta_{il}$ (see Varenna notes p.~106).

The $i$-wave eigevectors, $z_i$ in terms of the
eigenvectors of the $ll$ block of the hamiltonian, $z_l$ are, therefore,
$$z_i({\bf k})={1\over\sqrt{\Gamma^\beta_i}} S^\beta_{il}
      \sqrt{d_l} z_l({\bf k})$$
Note that, again, we need only use the tight-binding potential
parameters even in the $i$-set.
\vskip 10pt
\line{4. Accumulating the charge density\hfil}
\nobreak\vskip 5pt\nobreak
Having recovered eigenvectors in the $i$-channels one may proceed to
calculate moments of the spherical ASA charge density. Explicit
expressions are given in equations 2.66 and 2.67 in the Kanpur notes.
In the $\beta$-representation we have chosen, there is no coupling
of the moments between the $l$- and $i$-channels since matrix elements
of the first order hamiltonian, $h^\beta$ are zero in the $i$-channels.
Therefore the moments in the $l$ and $i$ channels may be accumulated
independently. Since the $i$-eigenvectors are only correct to zeroth
order, their first and second moments are, in fact, not defined in any
representation. However, in order to estimate new logarithmic
derivatives
(or $e_\nu$'s) for the next iteration, the first and second moments
are accumulated at each k-point from the eigenvalues of the hamiltonian.
Thus the moments in the $i$-channels are
$$m^0_i=\sum_{n,{\bf k}{\rm (occ.)}}\vert z_i({\bf k})\vert^2$$
$$m^r_i=\sum_{n,{\bf k}{\rm (occ.)}}
m^0_i (e_{n,{\bf k}} - e_\nu)^r\quad,\qquad r=1,2$$
In this way, charge is properly distributed into the $i$-waves and the
$e_\nu$ correctly shifted to the centres of the occupied bands during
the self-consistency procedure.
\end
