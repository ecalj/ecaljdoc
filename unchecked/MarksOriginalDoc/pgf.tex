\documentclass{article}
\usepackage[dvips]{graphicx}
\usepackage{html,makeidx,color}

\def\itemia{\addtocounter{enumi}{1}\item[\arabic{enumi}]}
\def\itemir{\addtocounter{enumi}{1}\item[\roman{enumi}]}
\def\itemiR{\addtocounter{enumi}{1}\item[\Roman{enumi}]}
\def\itemiia{\addtocounter{enumii}{1}\item[\arabic{enumii}]}
\def\itemiir{\addtocounter{enumii}{1}\item[\roman{enumii}]}
\def\itemiiR{\addtocounter{enumii}{1}\item[\Roman{enumii}]}

\begin{document}

\title{ASA layer Green's function package (v6.14)}
\author{Mark van Schilfgaarde}
\date{28 Oct, 2004}
\maketitle

\section{Introduction}
\label{sec:intro}

This package implements the ASA local spin-density approximation
using Green's functions (GF) in a special layer geometry.  It
adds a program `lmpg' to the ASA suite, which plays approximately
the same role as the LMTO-ASA band program `lm' or the crystal
Green's function program `lmgf'.

{\bf lmpg} is similar in most respects to the crystal GF package, except for
some significant complications that arise from treatment in a layer
geometry.  In {\bf lmpg}, there is a `special direction', which defines the
layer geometry, and for which the GF are generated in real space.  It
is specified by the third of the primitive lattice translation vectors
(token PLAT).  In the other two directions, Bloch sums are taken in
the usual way; thus for each qp in the parallel directions, the
hamiltonian becomes one-dimensional and is thus amenable to solution
in order-N time in the number of layers.

In the layer geometry, the material consists of an active, or
embedded region, which is cladded on the left and right by left
and right semi-infinite leads.  The crystal (or bicrystal) is
partitioned into slices, or principal layers (PL), in the
`special direction'.  This is done because the hamiltonian {\em
H} is short-ranged: if each PL is thick enough so that {\em H}
only connects neighboring PL, the computational effort scales
linearly with the number of PL.  Actually {\bf lmpg} just assumes
each PL is thick enough so that {\em H} is tridiagonal in the PL
representation.  The principal layers are defined by you in (see
token RMAX=, Sec.~\ref{sec:input}).  It is your responsibility to
see that each PL is thick enough so that {\em H} connect to only
nearest-neighbor PL on either side.

{\bf lmpg} has to deal with some of the complications arising
from how the endpoints are treated.  The first PL is treated as
the `left endpoint' and the last PL are treated as the `right
endpoint.'  The first PL and last PL are replicated to the left
and right, respectively, and extend outside of the embedding
region.  (Both are replicated implicitly an infinite number of
times to form the semi-infinite leads).  In the `left' case a
replica each site belonging to first PL is shifted `to the left'
by PLATL; similarly, a replica of each site belonging to last PL
are are shifted `to the right' by PLATR, so that we have

\[
  \ldots \,|\,
 \overbrace{\rm L-replica}^{\rm PLATL}\,|\,
 \overbrace{{\rm PL}\ 0 \,|\, {\rm PL}\ 1 \,|\, {\rm PL}\ 2 \,|\,  \ldots \,|\, {\rm PL}\ n{\rm{-}}1}^{\rm PLAT}\,|\,
 \overbrace{\rm R-replica}^{\rm PLATR}\,|\, \ldots
\]
See Sec.~\ref{sec:input} for how {\bf lmpg} reads PLAT, PLATL, PLATR.

It should be clear that the embedding region should be
constructed so that the PL adjacent to the leads are already
bulk-like.  The C- region should be made large enough so that
charge densities of the PL near the edges of the C- region
(e.g. PL 0 and 1, and PL {\em n}-1 and {\em n}) rather closely
resembles densities of the semi-infinite (bulk) leads.  This
point is discussed later.

As we discuss later, the GF of the embedding region is computed
from the (surface) GF of the L-replica and the (surface) GF of
the R-replica.  Thus space is partioned into three regions, the
left (L-) end region, the right (R-) end region, and the center,
embedding (C-) region.
The end regions are treated specially in two contexts:
\begin{enumerate}

  \item Surface Green's functions for the end regions are needed
to supply the boundary conditions for the embedding Green's
function

  \item {\bf lmpg} requires special treatment for the electrostatics joining
the three regions.

\end{enumerate}

\subsection{Green's functions for the end regions}
\label{sec:GFends}

By its construction the PL of each end (L- and R-) region would,
if repeated throughout all space, constitute a periodic solid.
{\bf lmpg} has a special branch (specified by token MODE=2,
Sec.~\ref{sec:input}) for the L- and R- PL that enable it to
generate the self-consistent left- and right- GF for each
corresponding periodic solid.  This is needed to make the
potential of each end region.  Note that it is assumed to be
bulk-like.  The GF should be the same as that generated by 3D
Green's function program {\bf lmgf}, except that this GF has a
mixed-real and {\em k}-space representation, and here separate
Green's functions and potentials are needed for each end region.
Also, owing to the mixed mixed-real and {\em k}-space
representation, the methodology for constructing {\em G} is
different.  We will call the periodic solid of repeating L- PL
the ``bulk'' crystal of the L- region; similarly for the R- PL.
Thus, there is a well-defined ``bulk'' GF and potential the L-
PL, and one also for the R- PL.

Once the L- and R- bulk potentials are made, the surface Green's
function can be computed, which is needed to supply the proper
boundary conditions to generate the GF in the embedding region.

\subsection{Self-consistency and charge neutrality}
\label{sec:neutrality}

A Green's function (or any band) method requires integration to
the Fermi level, which is determined by charge neutrality.  Also
in general, the electrostatic potential is determined up to an
arbitrary constant; but specifying the Fermi level determines the
constant, or vice-versa.  In
the band code, we supply (or assume) the constant, and find the
corresponding Fermi level.  In the GF programs we indirectly fix
this constant potential by specifying the Fermi level as an
input.  Requirement of charge neutrality fixes the constant
potential shift.  Note that the GF have a complication not
present in a band program, where the entire spectrum is computed:
the constant shift must be determined by an iterative procedure.
In the crystal GF case, it is iteratively determined by a Pade
approximant (see \htmladdnormallink{lmgf documentation}{gf.txt}).
Metals and nonmetals are distinguished in that in the latter
case, there is no DOS in the gap and therefore the Fermi level
(or potential shift, if the Fermi level is specified) cannot be
specified precisely.  As we will describe, something similar
happens in the layer case, but it is a little more complicated.
The Fermi level and constant potential are stored in array {\bf
vshft}, and permanently on disk in file {\bf vshft} described
below and in the documentation found in the source code {\bf
iovshf.f}.  Actually {\bf lmpg}'s freedom to shift potentials is
more general and can accomodate potential shifts at separate
sites, useful in non-self-consistent or limited self-consistent
calculations.

The layer GF case is complicated by the partioning into three
distinct regions.  Self-consistency, and therefore determination
of potential proceeds differently depending on whether one is
computing the potential for the bulk L- and R- (MODE=1) or for
the layer system (\dots\,$|$\,L\,$|$\,C\,$|$\,R\,$|$\,\dots)
(MODE=1).
Computing the bulk potential for the L- and R- regions (MODE=2),
is quite analogous to the crystal GF, albeit for two independent
regions L and R.  Periodic boundary conditions are assumed
separately; each case the end layer is assumed to be a periodic
solid and the (periodic) Green's function is computed for it.
Thus the potential in each end layer may be shifted independently
by a constant shift $V_L$ or $V_R$.  Self-consistency proceeds
analogously to the crystal GF, independently for the regions L
and R.  Constant potential shifts are determined independently
for each layer in the same way as the in crystal GF code; the
potential shift is computed that satisfies charge neutrality for
the corresponding periodic solids; see \htmladdnormallink{lmgf
documentation}{gf.txt}, for further details.  If the L- (R-) PL
is a metal, the constant potential shift is not adjustable,
because the Fermi level is specified at the outset.  Note,
however, that the Fermi level is only sharply defined for a
metal; thus it is important to distinguish metal and nonmetal
cases independently in the two layers.  You can set them with
tokens LMET= and RMET= in the PGF category; see
Sec.~\ref{sec:input}.  These tokens play the role of the METAL=
token for the crystal GF code.

Charge neutrality in the layer case
(\dots\,$|$\,L\,$|$\,C\,$|$\,R\,$|$\,\dots) is more complicated.
It should be satisfied independently in the each of the L-, C-
and R- regions.  In practice, we {\em assume} that the density in
the L- and R- end regions is bulk-like and does not change once
it has been calculated (MODE=2).  Thus changes in the density are
confined to the C- region.  The C- region has to be
charge-neutral because the L- and R- are already neutral, and the
entire (\dots\,$|$\,L\,$|$\,C\,$|$\,R\,$|$\,\dots) system must be
neutral.  The program proceeds by finding a shift that satisfies
charge-neutrality in the C- region and doesn't worry about the
rest.  This is reasonable since we assume at the outset that the
C- region has enough PL near L- and R- large enough to allow the
density to be bulk-like, no charge should spill into the L- and
R- regions by construction.  Thus, when computing the Green's
function of the (\dots\,$|$\,L\,$|$\,C\,$|$\,R\,$|$\,\dots)
system in practice, the layer code selects the shift $V_C$ so as
to eliminate the deviation from charge neutrality in the C-
region, following the method of the crystal code {\bf lmgf}.
Only sites in the C- region are shifted by $V_C$; the potentials
and charges in the L- and R regions are left untouched.  Once
$V_C$ is found and the corresponding Green's function is
generated, {\bf lmpg} returns to the sphere program where it
computes the potential functions for the updated sphere charges
and moments, computes the electrostatic potential (see below)
for the (\dots\,$|$\,L\,$|$\,C\,$|$\,R\,$|$\,\dots) system, and
begins another pass in the self-consistency cycle.

Because of deviations from self-consistency, and also because of
finite-size effects discussed above, there can be some deviation
from charge neutrality in the end regions.  {\bf lmpg} will
generate the GF in each end PL, so that the deviation from
neutrality may be computed. (You must have the 'bigemb' option
set for {\bf lmpg} to generate this information.)  The sphere
charges are shown in the table headed by the lines:
\begin{verbatim}
 PGFASA: integrated quantities from G
     PL      D(Ef)      N(Ef)       E_band      2nd mom      Q-Z
\end{verbatim}
and deviations of the end regions from charge neutrality are summarized
at the end of this table in lines similar to the following:
\begin{verbatim}
 Deviations from charge neutrality:
   Left end layer       0.003965
   Right end layer      0.003965
\end{verbatim}
However, {\bf lmpg} does not use this information; instead it
keeps the densities as computed for the bulk L- and R- crystals.
If these charges are not small, your active region should be
enlarged.

\subsection{Electrostatics}
\label{sec:electrostatics}

At self-consistency there is a unique potential defined by the
electrostatic potential from the charges at each site and a
global constant potential shift $V_C$ that shifts the entire
system.  This shift makes makes each region charge neutral and
possesses within the C- region a dipole that will exactly align
the Fermi levels in the L- and R- regions, as we now describe.

For now let's restrict ourselves to the case when both L- and R-
regions are metals.  We can form compute electrostatic potentials
in the L- and R- regions by two different constructions:

\begin{enumerate}

\item Electrostatic potentials in L- and R- may be computed as in
      MODE=2, that is by assuming L- and R- are bulk solids with
      periodic boundary conditions in the respective L- and R-
      regions.  In each case the potential is completely fixed by
      charge neutrality.

\item Electrostatic potentials in L- and R- may also be computed
      from solution of the potential of the entire
      (\dots\,$|$\,L\,$|$\,C\,$|$\,R\,$|$\,\dots) system.  This
      potential is adjustable up to some constant shift.

\end{enumerate}

In practice these two constructions produce different potential
in the L- and R- regions.  (Indeed, one might choose the constant
shift that ``best reconciles'' the mismatch in these two
constructions.  Versions of lmpg earlier than 6.14 did something
like this.  The present version choose the constant shift so as
to satisfy charge neutrality in the C- region, as was discussed
in the previous section.)

These potentials might be mismatched for two distinct reasons.
One error can arise because the density is not self-consistent.
More exactly the density doesn't possess the requisite dipole, so
that a global constant shift that ``best reconciles'' the
potential mismatch as constructed by the two methods above is
different from the one that ``best reconciles'' the mismatch on
the right.  Finite-size effects (a C- region with insufficient PL
near the end regions), is another source of error.

Recall that {\bf lmpg} freezes the potentials in the L- and R-
regions.  Thus, only the central region is affected by the shift
$V_C$ because the end PL shouldn't be shifted at all.  {\bf lmpg}
does print out the electostatic potentials computed by the two
different constructions, and summarizes the deviations in a
table similar to the following: {\tt\small
\begin{verbatim}
 Deviations in end potentials:
 region met  <ves>Bulk   <ves>layer     Diff      RMS diff
  L      T    0.055210    0.058192    0.002982    0.011265
  R      T    0.055210    0.058192    0.002982    0.011265

 RMS pot difference in L PL = 0.000127  in R PL = 0.000127  total = 0.000127
 vconst that minimizes potential mismatch to end layers =  0.058835
 vconst is now (estimate to satisfy charge neutrality)  =  0.061817
 difference                                             = -0.002982
\end{verbatim}

}

The top table compares the average electostatic potential in and
end region computed from the bulk geometry, and computed in the
(\dots\,$|$\,L\,$|$\,C\,$|$\,R\,$|$\,\dots) geometry.
The later numbers compare the potential that ``best reconciles''
the the methods of computation to the potential shift the program
will actually use.  Note that the potential shift you actually
should use is the one that meets charge-neutrality in the C-
region; indeed {\bf lmpg} will find this shift on its own if you
invoke it in self-consistent mode (section \ref{sec:input}).

When getting started, this table gives you a pretty reasonable
guess at the proper choice of the constant shift $V_C$.  That is,
you might set $V_C$ as the one that minimizes potential mismatch
to end layers.  You can adjust If you start out with the choice
$V_C$ by invoking {\bf lmpg} in an interactive mode, or by
setting file {\bf vshft} appropriately.  If you do so, you will
alleviate some of the burden on lmpg in determining this shift.

If the potential in either the L- or the R- differs significantly
(see case L- and R- are both metals, below), or if there is
significant deviation from charge neutrality in the end PL, the
user should enlarge the embedding region, as the end PL are not
sufficiently bulk-like.  This difference should vanish at
self-consistency if you construct the embedding region with PL
near the end regions similar to those of the last region.
Typically having 2 PL (say 0\dots{}1, and $n$-2\dots$n$-1 does a
pretty good job keeping the discrepancies between the bulk- and
layer potentials small.

Now we must distinguish between metal and nonmetal
cases.  If either end layer is a nonmetal, its potential can shift by
constant without affecting charge neutrality.  Therefore, if either L-
or R- regions is a nonmetal, there can be a constant shift in that
region (change in band offset).  Consider the following cases:

\begin{itemize}
\item

 L- and R- are both metals.  No potential shifts are allowed in the end
 layers.  In the self-consistency cycle (MODE=1) the program checks for
 deviations from charge neutrality, and adjusts the potential in the (C)
 region until neutrality is achieved.  However, the density resulting
 from this Green's function will generate charges and electrostatic
 potentials $V^i_m$ at sites $i$ in the L- and R- layers.  As mentioned
 above, potentials computed from the
 (\dots\,$|$\,L\,$|$\,C\,$|$\,R\,$|$\,\dots) system will reveal some
 differences relative to the electrostatic potentials $V^i_b$ when
 computed using just the L- or R- charges and a geometry for infinitely
 repeating L- and R- layers.  The potentials calculated these two ways
 is printed out, as described above.

\item

 Only one of L- or R- is a metal (LMET=f and RMET=t or vise-versa). Now
 the nonmetallic end region can shift its potential by a constant to
 best align to the potential computed from the
 (\dots\,$|$\,L\,$|$\,C\,$|$\,R\,$|$\,\dots) system.  We choose the
 constant in the nonmetallic PL that best aligns $V^i_b$ and $V^i_c$;
 that is that minimizes the RMS difference $V^i_b$ and $V^i_c$.

\item

 Neither the L- or R- is a metal (LMET=f and RMET=f).  If the C region
 is a metal (specified by BZ METAL=) the global potential shift should
 conform to the shift in the C region.  Both endpoints must be allowed
 to float (there are two distinct band offsets).  If no region is a
 metal, i.e. if there is no DOS at the Fermi level, the system should
 already be charge neutral and no shifts should be required.

\end{itemize}

\subsection{Electrostatics in layer geometry}
\label{sec:ewald}

The correct procedure to construct electrostatics in a layer
geometry is to carry out 2D Ewald summations for each PL, and add
up the contributions from all PL.  Because no one has come around
to making 2D Ewald sums yet for this program, we use a trick.

We compute the electrostatics via an Ewald summation of the
following supercell:

\[
 \overbrace{\rm L-replica}^{\rm PLATL}\,|\,
 \overbrace{\rm L-replica}^{\rm PLATL}\,|\,
 \overbrace{{\rm PL}\ 0 \,|\, {\rm PL}\ 1 \,|\, {\rm PL}\ 2 \,|\,  \ldots \,|\, {\rm PL}\ n{\rm{-}}1}^{\rm PLAT}\,|\,
 \overbrace{\rm R-replica}^{\rm PLATR}\,|\,
 \overbrace{\rm R-replica}^{\rm PLATR}\,|\,
\]

It assumes periodic boundary conditions of period in the layer direction
{\tt
\vskip 6pt
   PLAT(3) + 2*(PLATL + PLATR)
\vskip 6pt
}
\noindent
Note that this scheme is only approximate.  It will eventually be
replaced by 2D Ewald summations.

\section{Input for the layer Green's function program}
\label{sec:input}

Most of the input for {\bf lmpg} is similar to the band and crystal GF
programs.  This section describes input specific to {\bf lmpg}.

\vskip 12pt

\noindent
Each site must be assigned a `principal layer index' to tell {\bf lmpg}
which PL a site belongs to.  In category SITE, each site should
have a token PL:

{\tt
\vskip 6pt
  ATOM=`species-name'   PL=layer-index
\vskip 6pt
}

\noindent
Each group of sites with the same PL index are grouped together
into a single principal layer.

\noindent
Remember that the principal layers should be large enough such
that the range of the hamiltonian connects only adjacent PL.  The
range of the hamiltonian is fixed by the range of the structure;
it is set in the STR category, token RMAX= .

\noindent
There is a {\bf lmpg}-specific category, which includes the following:
{\tt
\vskip 6pt
   PGF    MODE=\#  PLATL=\# \# \#  PLATR=\# \# \#   GFOPTS= options   SPARSE=\#
\vskip 6pt
}

\noindent
{\bf Token} MODE= tells {\bf lmpg} what you want to do:

\begin{itemize}
\setcounter{enumii}{-1}
  \itemiia do nothing

  \itemiia calculate the diagonal GF, layer-by-layer.
	   This is the appropriate mode for self-consistent calculations

  \itemiia left- and right-bulk bulk GF.  The endpoints require
	   special treatment, and this mode is designed to generate
	   a self-consistent left- and right- bulk GF.   It should
	   be run before invoking {\bf lmpg} with MODE=1.


  \itemiia find k(E) for left bulk.  This uses a special trick
	   (see PRB 39, 923 (1989)) to find the wave numbers of
	   the left bulk GF corresponding to a given energy.

  \itemiia find k(E) for right bulk.

  \itemiia Calculate current using the Landauer formula.
\end{itemize}


\noindent
{\bf Tokens} specifying lattice vectors for the
(\dots\,$|$\,L\,$|$\,C\,$|$\,R\,$|$\,\dots) geometry:

\begin{tabular}{rl}
Tokens & PLATL=\# \# \#\\
and    & PLATR=\# \# \#\\
and    & a re-definition of PLAT(1:3,3)
\end{tabular}


\vskip 12pt
\noindent
The reason why PLATL and PLATR must be specified is because
{\bf{}lmpg} must have information about the semi-infinite
repeating layers that attach to each lead and extend to
$\pm{}\infty$.  Referring to the diagram below, there is implied
a second L-replica of PL 0 shifted relative to PL 0 by -2*PLATL,
another by -3*PLATL, and so on.  Similarly there is implied a
second R-replica shifted relative to PL $n$-1 by PLATR, another
by 2*PLATR, and so on.  This information is needed in order to
make the crystalline Green's functions (and also surface Green's
function) of each end layer.

\[
  \ldots \,|\,
 \overbrace{\rm L-replica}^{\rm PLATL}\,|\,
 \overbrace{{\rm PL}\ 0 \,|\, {\rm PL}\ 1 \,|\, {\rm PL}\ 2 \,|\,  \ldots \,|\, {\rm PL}\ n{\rm{-}}1}^{\rm PLAT}\,|\,
 \overbrace{\rm R-replica}^{\rm PLATR}\,|\, \ldots
\]

\vskip 12pt
\noindent
Caution: the inner product of PLATL with PLAT, and also the inner
product PLATR with PLAT must be positive.  That way each padding
layer adds to the length of PLAT(3).  (It is nonsensical for a
principal layer to have a negative thickness.)  Therefore, the
program will stop if either dot product is negative.

\vskip 12pt
\noindent
Note: at present, {\bf lmpg} calculates electrostatic potentials using
a 3D supercell approach; see section \ref{sec:ewald}.  It does so by creating periodic boundary
conditions in the third dimension with length
{\tt
\vskip 6pt
   PLAT(3) + 2*(PLATL + PLATR)
\vskip 6pt
}
\noindent
(Thus, PLATL and PLATR are needed in this second context as well,
to carry out the Ewald summations).  PLAT(1:3,3) printed as
output by program {\bf lmpg} reflects the addition of PLATL and
PLATR.
\vskip 12pt
\noindent
{\bf Token} GFOPTS={\em options-list}
\vskip 12pt
\noindent
specifies a collection of optional extra functions.  {\em options-list} is
a series of strings string1;string2;\dots
The following individual strings specify:
\begin{description}
  \item[emom]   generate the output ASA moments, needed for self-consistency

  \item[idos]   make integrated properties, such as the sum of one-electron energies

  \item[dmat]   make the density-matrix $G_{RL,R'L'}$

  \item[sdmat]  make the site-diagonal density-matrix $G_{RL,RL'}$
                The density matrix is written to a file `dmat'

  \item[pdos]   Make the partial density of states (this has never been checked)


  \item[p3]     Use third order potential functions
\end{description}

\vskip 12pt
\noindent
{\bf Token} SPARSE=1
\vskip 12pt
\noindent
uses a modified LU decomposition to generate
the layer GF.  It tends to be significantly faster than the usual
approach; compare test cases 5 and 6 in shell script pgf/test/test.pgf/

\subsection{Potential shifts}

File {\bf vshft} holds information about potential shifts (
sections ~\ref{sec:neutrality} and ~\ref{sec:electrostatics})
The global shifts are contained in the first line and keep
information about Fermi level, the global constant shift,
and the shifts at the L and R end regions needed to match the
Fermi level.  The syntax for the first line is
{\tt
\vskip 6pt
 ef=\# vconst=\# vconst(L)=\# vconst(R)=\#
\vskip 6pt
}

You can optionally add site-dependent potential shifts.  After
the first line, add a line {\tt site shifts} followed by as
many lines as desired, one line per site shift, e.g.:

{\tt
\vskip 6pt
\begin{verbatim}
 ef=.03 vconst=-.01 vconst(L)=.02 vconst(R)=.03
 site shifts
  3    .1
  4    .2
\end{verbatim}
\vskip 6pt
}

\section{Program operation}
\label{sec:operation}

{\bf lmpg} starts by creating the left surface, and then proceeds
`left-to-right' layer-by-layer to generate the surface GF 0,1,2,3,...
until the rightmost layer is reached.  At that point the right surface
GF is generated, and the crystal GF is generated by embedding between
the left- and right- surface GF.  Then using Dyson's equation, {\bf lmpg}
proceeds layer-by-layer `right-to-left' to generate the crystal GF
from the surface GF on the left and the crystal GF on the right.
This is done for each energy and k-point in the two-dimensional BZ.

\subsection{Use in conjunction with other programs}

You can use plane-analysis lmplan to analyze charge distributions
by plane, and also use it to create a `site' file to facilitate
making of lattices with periodic boundary conditions so that you
can run programs `lm' and `lmgf' for comparison.

At present {\bf lmpg} cannot make the static response function, as can {\bf lmgf};
this can vastly improve effiency for self-consistency.  However, if
you make the response function using {\bf lmgf}, you can use it with the
{\bf lmpg} program.  Here is an example taken from the directory
of pgf/tests/copt.

\vskip 12pt
\noindent
Invoke this command
{\tt
\vskip 6pt
lmplan copt -vpgf=1 -cstrx=file -vlmf=f -vnk1=6 --pr31 -vnit=10 -vgamma=f -vsparse=0 --no-iactiv --time=5
\vskip 6pt
}
\vskip 12pt
\noindent
At the prompt, type
{\tt
\vskip 6pt
  wsite -pad abc
\vskip 6pt
}

\vskip 12pt
\noindent
lmplan creates a site file named `abc.copt' suitable for use with programs lm
and {\bf lmpg}.  It uses the padding layers as buffer layers.  Conveniently,
it satisfies periodic boundary conditions.  To verify this, try

{\tt
\vskip 6pt
\noindent
> cp abc.copt site.copt \\
> lmchk copt -vpgf=0 -cstrx=file -vlmf=f -vnk1=6 --pr31 -vnit=10 -vgamma=f -vsparse=0 --no-iactiv --time=5
\vskip 6pt
}

Now you can make a suitable ASA static response function suitable for the
layer code with

{\tt
\vskip 6pt
> lmgf copt -vpgf=0 -cstrx=file -vlmf=f -vnk1=6 --pr31 -vnit=10 -vgamma=f -vsparse=0 --iactiv --time=5 -vscr=1
\vskip 6pt
}

\noindent
The {\em q=0} response function is written to file `psta.copt.'
(This file has already been created and sits in pgf/test/copt.)  Once psta is
created, it can greatly facilitate convergence to
self-consistency.  Try running, for example, the test script
{\tt
\vskip 6pt
\noindent
>  pgf/test/test.pgf --usepsta copt 5
\vskip 6pt
}
\noindent
will use the psta file to assist convergence when you use the switch
--usepsta.  Look in particular at the RMS DQ, viz:
{\tt
\vskip 6pt
\noindent
> grep RMS out.lmpg.self-consistent \\
 PQMIX:  read 0 iter from file mixm.  RMS DQ=1.84e-3 \\
 PQMIX:  read 1 iter from file mixm.  RMS DQ=1.59e-4  last it=1.84e-3 \\
 PQMIX:  read 2 iter from file mixm.  RMS DQ=3.30e-4  last it=1.59e-4 \\
 PQMIX:  read 3 iter from file mixm.  RMS DQ=4.66e-5  last it=3.30e-4 \\
\vskip 6pt
}
\noindent
If you invoke it in the usual way, viz without --usepsta:
{\tt
\vskip 6pt
\noindent
>  pgf/test/test.pgf copt 5
\vskip 6pt
}
\noindent you should see the following lines
{\tt
\vskip 6pt
\noindent
> grep RMS out.lmpg.self-consistent \\
 PQMIX:  read 0 iter from file mixm.  RMS DQ=1.44e-3 \\
 PQMIX:  read 1 iter from file mixm.  RMS DQ=6.41e-3  last it=1.44e-3 \\
 PQMIX:  read 2 iter from file mixm.  RMS DQ=2.22e-1  last it=6.41e-3 \\
 PQMIX:  read 3 iter from file mixm.  RMS DQ=3.05e-2  last it=2.22e-1 \\
 PQMIX:  read 4 iter from file mixm.  RMS DQ=3.01e-2  last it=3.05e-2 \\
 PQMIX:  read 5 iter from file mixm.  RMS DQ=2.66e-2  last it=3.01e-2 \\
 PQMIX:  read 6 iter from file mixm.  RMS DQ=4.15e-2  last it=2.66e-2 \\
 PQMIX:  read 7 iter from file mixm.  RMS DQ=4.92e-2  last it=4.15e-2 \\
 PQMIX:  read 8 iter from file mixm.  RMS DQ=4.40e-2  last it=4.92e-2 \\
 PQMIX:  read 8 iter from file mixm.  RMS DQ=2.24e-2  last it=4.40e-2 \\
\vskip 6pt
}
\noindent
The improvement with the response function is dramatic.

\vskip 12pt
\noindent
{\sl Test cases}

Shell script pgf/test/test.pgf checks out that the program is working
properly, and it also is convenient to illustrate some of the features
in {\bf lmpg}.


\end{document}

Nonequilibrium case:

  EMESH= nz mode emin emax e1 e2 nzne vne delne delse

  nz = number of energy points

  mode specifies the kind of contour; see below

  emin,emax are the energy window (emax is usually the Fermi level)

  mode=10: the energy mesh is a Gaussian quadrature on an ellipse
  e1 	is the eccentricity of the ellipse, ranging from 0 (circle) to 1 (line)
  e2 	is a 'bunching' parameter that, as made larger,
     	tends to bunch points near emax.  As a rule, e2=0 is good, or
     	maybe e2=.5 to emphasize points near Ef.

  nzne  number of mesh points along real axis, nonequilibrium contribution
 
  vne   potential drop across active region
 
  delne Imaginary part of energy for nonequilibrium contour

  delse Imaginary part of energy for self-energy
