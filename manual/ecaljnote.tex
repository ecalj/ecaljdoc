\documentclass[a4paper,10pt,epsf,fleqn]{article}
%#BIBTEX bibtex ecaljmanual
%\documentclass[a4paper]{jsarticle}
\setlength{\mathindent}{0mm}
%\usepackage[dvipdfm]{graphicx} 
%\usepackage{bmpsize}
\oddsidemargin=1cm\pagebreak[2]
\topmargin=-1cm
\setlength{\textwidth}{15cm}
\setlength{\textheight}{24cm}
\pagestyle{plain}
%
\usepackage{amsmath}
\usepackage{graphicx}	% required for `\includegraphics' (yatex added)
%\author{Takao Kotani}
%\title{ecalj --- Get statrted} 
\usepackage{setspace}
\usepackage{hyperref}
\usepackage{ulem}
\usepackage[usenames]{color}
\usepackage{makeidx}
\usepackage{ascmac}
%\bibliographystyle{apsrev4-1}
\bibliographystyle{unsrt}
\setstretch{1.1}

\def\psibar{\bar{\psi}}
\def\psidotbar{\dot{\bar{\psi}}}
\def\scgw{{sc{\em GW}}}
\def\G0{G^0}
\def\tphi{{\tilde{\phi}}}
\def\calR{{\cal A}}
\def\qsgw{QS{GW}}
\def\QSGW{QS{GW}}
\def\ldagw{{lda{\it GW}}}
\def\GLDA{{G^{\rm LDA}}}
\def\WLDA{{W^{\rm LDA}}}
\def\ekn{{\varepsilon_{{\bf k}n}}}
\def\phidot{\dot{\phi}}
\def\phidottilde{\dot{\tilde{\phi}}}
\def\phitilde{\tilde{{\phi}}}
\def\epsilonaone{\epsilon^{(1)}_a}
\def\epsilonatwo{\epsilon^{(2)}_a}
\def\ei{\varepsilon_i}
\def\eis{\varepsilon_{i\sigma}}
\def\ej{\varepsilon_j}
\def\Ekn{{E_{{\bf k}n}}}
\def\Psikn{\Psi_{{\bf k}n}}
\def\Psiqn{{\Psi_{{\bf q}n}}}
\def\Psiqm{{\Psi_{{\bf q}m}}}
\def\DVo{{\it \Delta}V(\omega)}
\def\DVoret{{\it \Delta}V^R(\omega)}
\def\DVoadv{{\it \Delta}V^A(\omega)}
\def\DV{{\it \Delta}V}
\def\DVhat{{\it \Delta}\hat{V}}
\def\HLDA{H^{\rm LDA}}
\def\veff{V^{\rm eff}}
\def\vxc{V^{\rm xc}}
\def\Dvxc{{\it \Delta} V^{\rm xc}}
\def\vc{V^{\rm c}}
\def\vext{V^{\rm ext}}
\def\hVext{\hat{V}^{\rm ext}}
\def\hVeff{\hat{V}^{\rm eff}}
\def\hvnl{\hat{V}^{\rm nl}}
\def\vnl{V^{\rm nl}}
\def\vh{V^{\rm H}}
\def\vgw{V^{GW}}
\def\ReDVo{ {\rm Re}[{\it \Delta}V(\omega)] }
\def\ImDVo{ {\rm Im}[{\it \Delta}V(\omega)] }
\def\gwa{$GW$\!A}
\def\hVee{\hat{V}^{\rm ee}}
\def\hVext{\hat{V}^{\rm ext}}
\def\hHk{\hat{H}^{\rm k}}
\def\Sigmax{{\Sigma}^{\rm x}}
\def\Sigmac{{\Sigma}^{\rm c}}
\def\Heff{\hat{H}^{\rm eff}}
\def\vbar{\bar{V}}
\def\Sbarc{\bar{\Sigma}^{\rm c}}
\def\scgw{{QS{\em GW}}}
\def\ldagw{{lda{\em GW}}}
\def\ekn{{\varepsilon_{{\bf k}n}}}
\def\ekm{{\varepsilon_{{\bf k}m}}}
\def\eknp{{\varepsilon_{{\bf k}n'}}}
\def\Ekn{{E_{{\bf k}n}}}
\def\Psiqkm{{\Psi_{{\bf q}-{\bf k}m}}}
\def\Psikn{{\Psi_{{\bf k}n}}}
\def\Psikm{{\Psi_{{\bf k}m}}}
\def\Psikmstar{{ \Psi_{{\bf k}m}^*} }
\def\Psiknp{{\Psi_{{\bf k}n'}}}
\def\Psiqnp{{\Psi_{{\bf q}n'}}}
\def\Psiqn{{\Psi_{{\bf q}n}}}
\def\brl{{\bf R}L}
\def\brlp{{{\bf R}'L'}}
\def\tili{{\widetilde{i}}}
\def\tilj{{\widetilde{j}}}
\def\tiln{{\widetilde{n}}}
\def\tilm{{\widetilde{m}}}
\def\eak{\varepsilon_{\rm a}(\bfk)}
\def\ebk{\varepsilon_{\rm b}(\bfk)}
\def\iDelta{{\it \Delta}}
\def\efermi{\mbox{$E_{\rm F}$}}
\def\we{\mbox{$\omega_\varepsilon$}}
\def\eal{\varepsilon_{al}}
\def\eallo{\varepsilon^{\rm Lo}_{al}}
\def\smh{smHankel}
\def\smhs{smHankels}
\def\shotone{OneShot}
\def\shotonez{OneShot Z=1}
\def\x{\mbox{$\times$}}
\def\xccut{ {\rm xccut} }
\def\xccutone{ {\rm xccut1} }
\def\xccuttwo{ {\rm xccut2} }
\def\ftn[#1]{\rlap{\footnotemark[#1]}}
\def\tr{{\rm Tr}}
\def\bQP{{\it bare QP}}
\def\bQPs{{\it bare QPs}}
\def\dQP{{\it dressed QP}}
\def\dQPs{{\it dressed QPs}}
\def\Re{{\rm Re}}
\def\EMAX{  E^{\rm APW}_{\rm MAX} }
\def\EMAXm{ E^{\rm rmesh}_{\rm MAX} }
\def\EMAXS{E_{\rm MAX}^\Sigma}
\def\NAPW{N_{\rm APW}}
\def\RSM{R_{\rm SM}}
\def\RSMa{R_{{\rm SM},a}}
\def\RSMal{R_{{\rm SM},al}}
\def\epsilonal{\epsilon_{al}}
\def\RGS{R_{\rm G}}
\def\RGSa{R_{{\rm G},a}}
\def\pakl{p_{akl}}
\def\PakL{P_{akL}}
\def\wPakL{\widetilde{P}_{akL}}
\def\CakL{C_{akL}}
\def\CiakL{C^i_{akL}}
\def\EMAX{  E^{\rm APW}_{\rm MAX} }
\def\EMAXm{ E^{\rm rmesh}_{\rm MAX} }
\def\EMAXvxc{ E^{\rm vxc}_{\rm MAX} }
\def\nc{n^{\rm c}}
\def\nzc{n^{\rm Zc}}
\def\nzcv{n^{\rm Zcv}}
\def\barnzcv{\bar{n}^{\rm Zcv}}
\def\MM{{\cal M}}
\def\RR{v}
\def\inta{\int_{|\bfr|\leq R_a}\!\!\!\!\!\!\!\!\!\!\!\!}
\def\intaa{\int_{|\bfr|\leq R_a}}
\def\intad{\int_{|\bfr'|\leq R_a}\!\!\!\!\!\!\!\!\!\!\!\!}
\def\intar{\int_{|\bfr-\bfR_a|\leq R_a}}
\def\intard{\int_{|\bfr'-\bfR_a|\leq R_a}}
\def\rhoij{\rho_{ij}}
\def\ekcore{E_{\rm k}^{\rm core}}
\def\ek{E_{\rm k}}
\def\ehf{E_{\rm Harris}}
\def\nin{n^{\rm in}}
\def\nout{n^{\rm out}}
\def\Vin{V}
\def\iDelta{{\it \Delta}}
\def\philo{{\phi}^{\rm Lo}_{al}}
\def\DEe{{\it \Delta} E_{\rm e}}
\def\ERPA{E^{\rm RPA}}
\def\bfp{{\bf p}}
\def\bfP{{\bf P}}
\def\EMAX{  E^{\rm APW}_{\rm MAX} }
\def\H0{H^0}
\def\hHZ{\hat{H}^0}
\def\hH{\hat{H}}
\def\underconstruction{{\it...xxxxx under construction xxxxx...\\}}

\newcommand{\fl}[1]{\noindent{\sf $\bullet$ #1\index{\sf #1}} : }
\newcommand{\fx}[1]{\subsection{\sf #1\index{\sf #1}}}
\newcommand{\ssx}[1]{\subsection{\bf #1\index{\bf #1}}}
\newcommand{\ssxx}[2]{\subsection{\bf #1\index{\bf #2}}}
\newcommand{\infiles}{\noindent\fbox{Input files}}
\newcommand{\outfiles}{\noindent\fbox{Output files}}
\newcommand{\GW}{$GW$}
\newcommand{\GWinput}{{\sf GWinput}\ }
\newcommand{\GWIN}{{\sf GWIN}\ }
\newcommand{\gbox}[1]{\noindent{\color{Green}\fbox{\parbox{260mm}{#1}}}}
\newcommand{\rbox}[1]{\noindent{\color{Red}\fbox{\parbox{260mm}{#1}}}}
\newcommand{\obox}[1]{\noindent{\color{Orange}\fbox{\parbox{260mm}{#1}}}}
\newcommand{\cyanbox}[1]{\noindent{\color{Cyan}\fbox{\parbox{260mm}{#1}}}}
\newcommand{\bluebox}[1]{\noindent{\color{Blue}\fbox{\parbox{260mm}{#1}}}}
\newcommand{\keyw}[1]{\fbox{\tt #1}}
\newcommand{\innera}[1]{{[\![#1]\!]_{R_a}}}
\newcommand{\bfzero}{{\bf 0}}
\newcommand{\bfq}{{\bf q}}
\newcommand{\bfk}{{\bf k}}
\newcommand{\bfr}{{\bf r}}
\newcommand{\bfX}{{\bf X}}
\newcommand{\hbfr}{\hat{\bf r}}
\newcommand{\bfQ}{{\bf Q}}
\newcommand{\bfT}{{\bf T}}
\newcommand{\bfG}{{\bf G}}
\newcommand{\bfR}{{\bf R}}
\newcommand{\ds}{\displaystyle}
\newcommand{\exe}[1]{{\tt #1}}
\newcommand{\io}[1]{{\sf  #1}}
\newcommand{\raw}[1]{{\tt #1}}
\newcommand{\repp}[1]{p.\pageref{#1}}
\newcommand{\refeq}[1]{Eq.~(\ref{#1})}
\newcommand{\reffig}[1]{Fig.\ref{#1}}
\newcommand{\smH}{{\mathcal H}}
\newcommand{\YY}{{\cal Y}}
\newcommand{\GG}{{\cal G}}
\newcounter{Alist}
\newcommand{\ul}[1]{\underline{#1}}
\newcommand{\ocite}[1]{\cite{#1}}
\newcommand{\ispone}{}
\newcommand{\isptwo}{}
\newcommand{\ooplus}{\oplus}
\newcommand{\oominus}{\ominus}

\newcommand{\req}[1]{\mbox{Eq.~(\ref{#1})}}
\newcommand{\refsec}[1]{\mbox{Sec.~\ref{#1}}}

\newcommand{\hHzero}{\hat{H}^{0}}
\newcommand{\CikL}{{C^{(i)}_{kL}}}
\newcommand{\CiRkL}{{C^{(i)}_{{\bf R}kL}}}
\newcommand{\tPkL}{{\widetilde{P}_{kL}}}
\newcommand{\tPRkL}{{\widetilde{P}_{{\bf R}kL}}}
\newcommand{\PkL}{{P_{kL}}}
\newcommand{\PRkL}{{P_{{\bf R}kL}}}
\newcommand{\rmt}{{s_{\bf R}}}
\newcommand{\val}{{\rm{VAL}}}
\newcommand{\core}{{\rm{CORE}}}
\newcommand{\xc}{_{\rm{xc}}}
\newcommand{\CORE}{{CORE}}
\newcommand{\COREone}{{CORE1}}
\newcommand{\COREtwo}{{CORE2}}
\newcommand{\VAL}{{\rm{VAL}}}
\newcommand{\EF}{E_{\rm F}}
\newcommand{\oneshotgw}{1shot-$GW$}
\newcommand{\incg}[1]{\includegraphics[width=5.9cm]{#1}}
\newcommand{\incgg}[1]{\includegraphics[width=8.5cm]{#1}}
\newcommand{\bfe}{{\bf e}}
\newcommand{\eiqr}{e^{i \bfq \bfr}}
\newcommand{\figp}[1]{\rotatebox{-90}{\includegraphics[width=10cm]{#1}}}
\newcommand{\bfex}{{\bf e}_x}
\newcommand{\bfey}{{\bf e}_y}
\newcommand{\bfez}{{\bf e}_z}
\newcommand{\bfa}{{\bf a}}
\newcommand{\bfb}{{\bf b}}
\newcommand{\bfS}{{\bf S}}
\newcommand{\bfiS}{{\it \Delta \bf S}}
\newcommand{\bfB}{{\bf B}}
\newcommand{\eps}{\epsilon}
\newcommand{\D}{{\it \Delta}}
\newcommand{\figss}[2]{\hspace{-3cm}\rotatebox{-90}{\includegraphics[width=6cm]{#1}}\rotatebox{-90}{\includegraphics[width=6cm]{#2}}}
\newcommand{\figs}[2]{\hspace{-2cm}\rotatebox{0}{\includegraphics[width=8cm]{#1}}\rotatebox{0}{\includegraphics[width=8cm]{#2}}}
\newcommand{\questaal}{{\tt Questaal}\ }
\newcommand{\ecalj}{{\tt ecalj}\ }
\newcommand{\lmchk}{{\exe{lmchk}\space }}
\newcommand{\ctrl}{{\io{ctrl.*}\space }}
\newcommand{\ctrls}{{\io{ctrls.*}\space }}
\newcommand{\alat}{{\raw{alat}}}
\newcommand{\plat}{{\raw{plat}}}
\newcommand{\qlat}{{\raw{qlat}}}

\def\ehf{E_{\rm Harris}}
\def\ehk{E_{\rm HoKohn}}

%\newenvironment{mspace0}{\baselineskip=1mm}
%\newenvironment{mspace}{\baselineskip=2mm}


\makeindex

\begin{document}
\baselineskip=6mm
\title{ ecalj note}
\author{\url{https://github.com/tkotani/ecalj}}
%\date{April 2nd 2001}
\maketitle

This note becomes rather obsolete.
I have to revise things step by step.

% \section{Product basis}
% This section is for the product basis to expand $v$ and $W$ within MTs.

% \hrule  width 15cm
% {\baselineskip=2.6mm
% \begin{verbatim}
% <PRODUCT_BASIS>
%  tolerance to remove products due to poor linear-independency
%   0.100000D-04 ! =tolopt; larger gives smaller num. of product basis. See lbas and lbasC, which are output of hbasfp0.
%  lcutmx(atom) = maximum l-cutoff for the product basis.  =4 is required for atoms with valence d, like Ni Ga
%   4  3
%   atom   l  nnvv  nnc ! nnvv: num. of radial functions (valence) on the augmentation-waves, nnc: num. for core.
%     1    0    2    3
%     1    1    2    2
%     1    2    3    0
%     1    3    2    0
%     1    4    2    0
%     2    0    2    1
%     2    1    2    0
%     2    2    2    0
%     2    3    2    0
%     2    4    2    0
%   atom   l    n  occ unocc  ! Valence(1=yes,0=no) 
%     1    0    1    1    1   ! 4S_p  ----- 
%     1    0    2    1    0   ! 4S_d        
%     1    1    1    1    1   ! 4P_p        
%     1    1    2    0    0   ! 4P_d        
%     1    2    1    1    1   ! 4D_p        
%     1    2    2    0    0   ! 4D_d        
%     1    2    3    1    1   ! 3D_l        
%     1    3    1    0    1   ! 4f_p        
%     1    3    2    0    0   ! 4f_d        
%     1    4    1    0    0   ! 5g_p        
%     1    4    2    0    0   ! 5g_d        
%     2    0    1    1    1   ! 2S_p  ----- 
%     2    0    2    0    0   ! 2S_d        
%     2    1    1    1    1   ! 2P_p        
%     2    1    2    0    0   ! 2P_d        
%     2    2    1    1    1   ! 3d_p        
%     2    2    2    0    0   ! 3d_d        
%     2    3    1    0    1   ! 4f_p        
%     2    3    2    0    0   ! 4f_d        
%     2    4    1    0    0   ! 5g_p        
%     2    4    2    0    0   ! 5g_d        
%   atom   l    n  occ unocc  ForX0 ForSxc ! Core (1=yes, 0=no)
%     1    0    1    0    0      0    0    ! 1S -----
%     1    0    2    0    0      0    0    ! 2S      
%     1    0    3    0    0      0    0    ! 3S      
%     1    1    1    0    0      0    0    ! 2P      
%     1    1    2    0    0      0    0    ! 3P      
%     2    0    1    0    0      0    0    ! 1S -----
% </PRODUCT_BASIS>
% \end{verbatim}}
% \hrule width 15cm
% \vspace{5mm}
% \noindent This section is read in the free format in fortran.
% So, e.g., \verb#0.01# works as same as \verb#0.10000D-01#.
% The line order is important 
% (you have to keep the order given by \io{GWinput.tmp}).
% Be careful atom atom id---lmf may re-order it and pass it to gw code.
% Look into LMTO file (generated by {\bf mkGWIN\_lmf2}); 
% which contains crystal structure information after such re-ordering by lmf.
% I used \verb#!# to make  clear that things after \verb#!#
% are comments. But \verb#!# is not meaningful -- just the expected
% numbers of data separated by blank(s) are read for each line 
% from the beginning of lines.

% \begin{itemize}
% \item
% \verb+0.100000D-02 ! =tolopt+ controls a number of Product basis
% to expand the Coulomb interaction within MTs. 
% \raw{tolopt} is a criterion to remove the poorly linear-independent product basis.
% Note that the product basis, which is to expand the
% Coulomb interaction, is different from the basis to expand eigenfunctions.
% In our experience, \verb+0.100000D-02+ (=0.001) is not so bad.
% If you like to reduce computational time use 0.01 or so, but a little
% dangerous in cases. With 0.0001, we can check stability on it.\\
% (note: By supplying multiple numbers, we can specify \raw{tolopt} atom by atom.
%  Remember \exe{lmchk} gives atom ID.)

% \item
%      lcutmx(atom) is the l cutoff of product basis for atoms 
%      in the primitive cell (do lmchk for atom id).
%      In the case of Oxygen, we can usually use lcutmx=2 (need check by
%      the difference when you use lcutmx=2 or lcutmx=4). 
%      Then the computational time is reduced well.

% \item
% (dec2014:\verb#<PBASMAX># is not checked recently;
% see \verb#fpgw/main/hbasfp0.m.F# and \verb#fpgw/gwsrc/basnfp.F#).)
% You can use \verb#<PBASMAX># section to override this setting. It is given as
% \begin{verbatim}
% <PBASMAX>
% 1  5 5 5 3 3
% 2  5 5 3 2 3
% 3  3 3 2 2 2
% </PBASMAX>
% \end{verbatim}
% The first number is for atom index (fixed), and other are product basis 
% for each $l$ channel.


% \item
% The integer numbers in 4th line \raw{lcutmx}
% gives the maximum angular momentum $l$ for the product basis
% for each atomic site.
% In our experience, \raw{lcutmx}=4 is required
% when the semi-core (or valence ) $3d$ electrons exist
% and we want to calculate the QP energies of them.

% \item
% Keep a block starting from 
% "  atom   l  nnvv  nnc ..."  as it originally generated 
% in \io{GWinput.tmp}. It just shows that how many kinds of radial functions
% for cores and valence electrons for each atom and l.
% {\tt nnvv}=2 in the case of $\phi$ and $\dot{\phi}$;
% {\tt nnvv}=3 in the case to add the local orbital in addition.

% \item
% There are two blocks after the line
% "{\tt   atom   l    n  occ  unocc  :Valence(1=yes, 0=no)}'
% and after
% "{\tt   atom   l    n  occ unocc  ForX0 ForSxc ! Core (1=yes, 0=no)}'.
% These are used to choose atomic basis to construct the product basis.
% The product basis are generated from the products of two atomic basis.

% {\sf GWinput.tmp} generated by {\bf mkGWIN\_lmf2} contains
% labels on each orbitals as \verb#4S_p#, \verb#4S_d#, \verb#4P_p#...
% Here \verb#4S_p# is for $\phi_{4s}$; \verb#4S_d# for $\dot{\phi}_{4s}$;
% \verb#3D_l# for $\phi^{\rm local}_{3d}$. 
% Capital letter just after the principle-quantum number
% means the orbital is used as `Head of MTO'; lowercase means just used only 
% as the `tail of MTO'.

% The switches for columns labeled as \verb#occ# and \verb#unocc#. take 0 (not included) 
% or 1 (included). With the switch, we can construct two groups of orbitals,
% \verb#occ# and \verb#unocc#.
% In this sample {\sf GWIN\_V2} as for atom 1,
% $\{ \phi_{4s},\dot{\phi}_{4s},\phi_{4p},\phi_{4d},\phi^{\rm local}_{3d}, 
% \phi^{\rm core}_{3s},\phi^{\rm core}_{3p} \}$
% consist the group \verb#occ#, and
% $\{ \phi_{4s},\phi_{4p},\phi_{4d},\phi^{\rm local}_{3d},\phi_{4f} \}$
% consists the group  \verb#unocc#.
% So the any product of combinations
% $\{ \phi_{4s},\dot{\phi}_{4s},\phi_{4p},\phi_{4d},\phi^{\rm local}_{3d}, 
% \phi^{\rm core}_{3s},\phi^{\rm core}_{3p} \}
% \times \{ \phi_{4s},\phi_{4p},\phi_{4d},\phi^{\rm local}_{3d},\phi_{4f} \}$
% are included as for the basis of the product basis.
% As for atom 2,
% $\{ \phi_{2s},\phi_{2p},\phi_{3d} \} 
% \times \{ \phi_{2s},\phi_{2p},\phi_{3d},\phi_{4f} \}$
% are included.


% \item
% Core section: (not worth to read, since we currently use no CORE2, \verb#A=B=C=0#.)\\

% \noindent Each line of the last section of {\tt Product BASIS} forms
% {\baselineskip=2.6mm
% \begin{verbatim}
%   atom   l    n  occ unocc   ForX0 ForSxc :CoreState(1=yes, 0=no)
%     1    2    1    A    x      B    C
% \end{verbatim}}
% At first you have to understand the concept of CORE1 and CORE2 in EQ.35 Ref.I.
% However, in our recent calculations, we do not use ``CORE2'' generally.
% So, in such a case, set \verb#A=B=C=0#. And treat shallow cores (above Efermi$-$2Ry or so )
% as valence electron by ``local orbital method'' in lmf.

% \item
%      Be careful. Current version is inconvenient...
%      Need to repeat mkGWIN\_lmf2 to generate GWinput template when you
%      add PZ (local orbital).
% \end{itemize}

% %As for the case of {\bf gwnc\_{\it foo}}, this section is neglected.
% \begin{quote}
% [( Note: you can skip here if you don't use CORE2.)

% Each of \raw{A,x,B,C} takes 0 or 1.
% There are some possible combination of these switches;
% \begin{enumerate}
% \item 
% If you take 
% {\tt ( A  x   B    C )= (1 0 1 1)},
% then the core is included in core2. In other words, this core is treated in the same 
% manner of the valence electron.

% \item 
% If you take
% {\tt ( A  x   B    C )= (0 0 0 0)},
% then the core is included in core1.
% The (exchange only) self-energy related to this core is included in {\tt SEXcore}.

% \raw{C} is the key switch which determine whether it is included in core1 or core2.
% There could be another option.

% \item 
% If you take
% {\tt ( A  x   B    C )= (1 0 0 1)}.
% This core is in core2. But it is not included in the calculation of $D$ and $W$.
% This core is only included for SEX and SEC calculations.
% \end{enumerate}
% These three kinds of choices are reasonable ones but we can consider some another choice.
% In the following, we show how these switches (\verb#A,B,C#) affect executions called from 
% \exe{gw\_lmfh} (essentially as same as \exe{gw\_lmf}).
% \begin{itemize}
% \item 
% \exe{hbasfp0}(mode 3) :Product basis for exchange due to core.\\
% We include the \raw{C}=0 cores as a part of
% the product basis as if \raw{A}=1 \raw{x}=0.

% \item 
% \exe{hsfp0}(mode 3): exchange mode for core.\\
% $\Sigma_{\rm x}$ only due to the \raw{C}=0 cores are calculated.

% \item 
% \exe{hbasfp0} (mode1): Product basis.\\
% Only see the switch \raw{A} and \raw{x}.
% The product basis is generated from (occupied $\times$ unoccupied), 
% where \raw{A}=1 core is included as one of the occupied basis.

% \item 
% \exe{hsfp0} (mode 1): exchange mode.\\
% Only see the switch \raw{C}.
% $\Sigma_{\rm x}$ due to valence and due to \raw{C}=1 cores
% are calculated.

% \item 
% \exe{hx0fp} (mode 1): $W-v$ calculation.\\
% Only see the switch \raw{B}.
% $W$ is calculates using all the valence and \raw{B}=1 cores.

% \item 
% \exe{hsfp0} (mode 2): correlation mode.\\
% Only see the switch \raw{C}.
% $\Sigma_{\rm c}$ due to valence and due to \raw{C}=1 are calculated.

% \end{itemize}

% \end{quote}

% \noindent $\bullet$ After you perform \verb#gw_lmfh# or anything,
% you find output files \io{lbas} by \exe{hbasfp0} (mode1), and/or \io{lbasc}
% by \exe{hbasfp0} (mode3) for core. These contains important information
% about how many and how product basis are chosen. 
% E.g. '\verb#grep nbloch lbas#' shows how many product basis are used in the calculations.



%\begin{verbatim}
% 1.Bandplot for FSMOMMETHOD/=0 
%   Even when you use FSMMOMMETHOD/=0 in GWinput for gwsc, 
%   you need to set FSMOMMETHOD=0 (or comment it out) when you run job_band_nspin2.
%   [If you run job_band_nspin2 with FSMOMMETHOD/=0, it make a shift 
%   (adding bias magnetic field).]

% 2.Note that ctrlgenM1.py automatically set this for --systype=molecule.
%    Then we have 
%        TETRA=0
%        N=-1  #Negative is the Fermi distribution function W= gives temperature.
%        W=0.001 #W=0.001 corresponds to T=157K as shown in console
%    In addiiton, FSMOM (n_up-n_down) is needed (FSMOMMETHOD=1)if we
%    have magnetic moment.

% 3. core>evalence message.
%    Ecore is grater than Evalence.
%    For save, we do not allow this.
%    Complare ECORE file and valence levels, shown in log file or
%    console output.

% 4. If you see a error message from lmf (e.g., internally called in the gwsc script).;
%   Exit -1 rdsigm: Bloch sum deviates more than allowed tolerance (tol=5e-6)
%   You have to enlarge RSRNG so that lmf finsh normally.

% 5. Back ground charge and fractional Z.
%    You can use fractional numbers for ATOM_Z, and also can set
%    valence charge by BZ_ZBAK (I removed BZ_VAL).
%    You see console out put, e.g,
%      "Charges:  valence    19.80000   cores     8.00000   nucleii   -28.00000
%         hom background       .20000   deviation from neutrality:      0.00000
%    . This is a case with BZ_ZBAK=.2.

%    NOTE: at the first iteration, Charges: shows such as
%      Charges:  valence     8.00000   cores    20.00000   nucleii   -28.00000
%       hom background     0.12300   deviation from neutrality: 0.12300
%       because of the initial condition by superposition of atoms. It show
%       deviation seems nonzero. But charge should be conserved from the
%       next iteration.

% 6. Not converged in metal. --->mixing may help
%    For example, if you try metal such as Bi2Sr2CuO6, it may fail at LDA/GGA level.
%    Then use ITER MIX=A2,b=.2. or something (.2 means it only mix 20% of output to give
%   new input for next iteration). Then I see convergence. (b is the
%    mixing parameter.

% 7. Use PZ or not.
%    If spillout of core is not so small (more then 0.05 or something.),
%    it is better to use PZ(lo). Treat the core as valecne.
%    Bi4d is such a case. Maybe use PZ=0,0,4.9

% 8. Core treatment 
%    See 10.1103/PhysRevB.76.165106 (Eq.35 and after).
%    Now I usually not use CORE2 (CORE1 only).

% 9. ERROR EXIT! rgwina: 2nd wrong l valence
%    This may be because you use wrong GWinput.
%    Back it up. And run mkGWIN_lmf2 (any n1 n2 n3 is fine).

% 10. Known bug.
%     Error occurs when system is anisotroic such as CuAlTe2. 
%     Temporary fix is "Add token NPWPAD=100 in HAM category".
%     (guess of used APW fails (more than expected)).
%     CuAlTe2,CuGaTe2 cases.

% 11. Known bug
%     a little unstable when metal GGA, especially when we have large empty regions.
% \end{verbatim}

% \section{Cautions for usage}
% \label{cautionusage}
% \begin{enumerate}
%  \item == not meaningful total energy in QSGW===\\
%  Total energy shown in QSGW mode in current version is not meaningful. 
% (just treat as an indicator to convergence).

% \item == Do we use VWN or GGA for QSGW? ===\\
%   In principle, QSGW results should not depends on VWN or GGA 
%   (XCFUN=1 or 103 in ctrl). But there is minor dependence, because\\
%    1. frozen core density.\\
%    2. core eigenfunctions.\\
%    3. radial basis functions\\
%    4. Slight numerical reason \\
%       (This is probably because Sigma-interpolation procedure
%        But not exactly figured out yet
%        $\rightarrow$ affect about 0.02eV as for band gap for GaAs. ).
%   In anyway, use VWN (HAM\_XCFUN=1) as standard.
%   And such technical things affects, 0.05 eV level of error for band gap.

% \item{EH and EH2} : For si, if EH and EH2 are the same, the following
%      error occurred. 
% \begin{verbatim}
%  fexit,fexit2,fexit3 error retval=          -1
%  Exit -1 zhev_tk2: nev /=nevx something wrong.
% \end{verbatim}
%      The large EH, EH2 get to be meaningless. We usually use up to 
%      $\sim 2$. (If you use very large EH such as E$\sim 10$, I am not so
%      sure weather it is )

%\item { The options about the product basis within MT. (SeungWoo's memo)}

% \begin{verbatim}
% <PRODUCT_BASIS>
%  tolerance to remove products due to poor linear-independency
%   0.100000D-02 ! =tolopt
% \end{verbatim}
% When the product basis are made, we may have poorly linear independent
% basis. For example, one in the set \{$f_1, f_2, ..., f_n$\} would 
% be almost give by a linear-combination of others. We need to make the
%       linear-independent set. Therefore, after calculating the overlap matrix 
%       $\langle  f_i|f_j \rangle$. We do diagonalization, then 
%       we remove eigenvectors corresponding to small eigenvalues than
%       \verb+tolopt+. 
%       See the {\bf hbasfp0} command in {\bf gwsc}

% \begin{verbatim}
% lcutmx(atom) = maximum l-cutoff for the product basis.
%   4  4  4  2  2  4  4
% \end{verbatim}
% For $\phi_1 \times \phi_2$ case, $|l_1 - l_2| \le l_{tot} \le |l_1 + l_2|$. 
% So `lcutmx' changes the maximum cutoff for the l$_{tot}$. The order is the same as the order of atoms in the {\bf ctrl} file.

% \begin{verbatim}
%   atom   l  nnvv  nnc !
%     1    0    3    3
%     1    1    3    2
%     1    2    2    1
% \end{verbatim}
% `atom' means the atom number identified in the {\bf ctrl} file.\\ 
% `l' is the angular momentum quantum number.\\
% `nnvv' is the number of radial functions (valence) on the augmentation-waves.\\
% `nnc' is the number of radial functions for core.\\
% The latter two ones, `nnvv' and `nnc', will be understood more clearly if you see the following ones.


% \begin{verbatim}
%   atom   l    n  occ unocc  ! Valence(1=yes,0=no)
%     1    0    1    1    1   ! 5S_p  -----
%     1    0    2    0    0   ! 5S_d
%     1    0    3    1    1   ! 4S_l
%     1    1    1    1    1   ! 5p_p
%     1    1    2    0    0   ! 5p_d
%     1    1    3    1    1   ! 4p_l
% \end{verbatim}
% Above options are about the product basis set within MT (Valence).\\
% `atom' and `l' are explained above. `nnvv' for `atom = 1 and l = 0' was `3' so this case we have 3 basis (`n = 1, 2, 3')\\
% `n' is the degree of freedom of the radial function, $\phi$. `n = 1'
%       means $\phi$, `n = 2' means $\dot\phi$, and `n = 3' means kind of
%       $\ddot\phi$, which the dot above the letter represents the
%       differentiation with respect to the energy. So `n = 1 and 2' is
%       related to the linearization of the radial function and `n = 3' is
%       the local orbital which is restricted in MT. The local orbital can
%       be modified changing `PZ' in the {\bf ctrl} file. Finally, the
%       number of the basis set which is needed for expanding eigenfunctions is $(l+1)^2 \times n$.\\
% `occ' and `unocc` mean that we use only ones that checked as `1', in other words we neglects `0' cases for making product basis. Be careful for confusion with name `occ' and `unocc'. These don't mean that occupied or unocc. When making product basis, $M = \phi_1 \times \phi_2$, `occ' corresponds to $\phi_1$ and `unocc' to $\phi_2$. For example, 
% \begin{verbatim}
%   atom   l    n  occ unocc  ! Valence(1=yes,0=no)
%     1    0    1    1    1   ! 5S_p  -----
%     2    3    1    0    1   ! 4f_p
% \end{verbatim}
% If the options are like the above, the product basis will be consists of ($\phi_1 = \phi_{atom=1,l=0}) \times (\phi_2 = \phi_{atom=1,l=0})$, ($\phi_{atom=1,l=0} \times \phi_{atom=2,l=3}$). As you can see, ($\phi_1 = \phi_{atom=2,l=3}$) is skipped.\\
% In the {\bf ctrl} file, `EH' controls the l part. As for `EH', (s, p, d, f) are used but {\bf GWinput} file uses (s, p, d, f, g). `EH' : HEAD part. `GWinput' : contains TAIL part... need more explanation.

% \begin{verbatim}
%   atom   l    n  occ unocc  ForX0 ForSxc ! Core (1=yes, 0=no)
%     1    0    1    0    0      0    0    ! 1S -----
%     1    0    2    0    0      0    0    ! 2S
%     1    0    3    0    0      0    0    ! 3S
% \end{verbatim}
% Above options are about the product basis set within MT (Core).\\
% `nnc' for `atom = 1 and l = 0' was `3' so this case we have 3 basis (`n = 1, 2, 3')\\
% Finally, for the convergence check, we can modify the following three things, (i) tolerance, (ii) lcutmx, and (iii) occ and unoccu.


%\end{enumerate}

% == Restart calculation in lda ==
%  lmf(lmf-MPIK) read rst.* in defaluts.
%  rst contains electron density.
%  If rst is already converged, it stops after two iteration.
%  rst contains atomic positions.
%  So, in order to read atomic positions change in ctrl,
%  Use options shown in lmf --help.

% == Restart calculation in qsgw ==
%   To remove mixsigm* (mixing for sigm), maybe required.


% == iteration check ===
%   First, watch console output of gwsc (do redirect to output file)
%   Need to check OK! signs arrayed on 1st columns.

%   gwsc iteration is time cosuming,
%   So we need to check calculations are normally going on or not.

%   Memory inefficiency.
%   Set 'KeepEigen off' an 'KeepPPOVL' off.
%   In fact, out code is still inefficient for memory usage.
  
%   grep gap llmf ---> minimum gap at mesh point.
%   see save.* ,or grep '[xc] ' save.*
%   the end of iteration of lmf is shown as x or c.
%   (if failed, QPU file. 
%   dqpu QPU.4run QPU.3run
%   As for usual semi-conductor, accuracy abou t0.1 eV is limit of current implementation.
%   Set vwn (xcfun=1) looks better (stable) for GW.

%   $grep rms lqpe* 
%   shows
%            ...  rmsdel=2.44D-04
%            ...  rmsdel=4.91D-03
%            ...   rmsdel=2.44D-04
%            ...   rmsdel=3.37D-04
%   If rsmdel is getteing to be smaller, it is on convergence path.
%   (but in magnetic cases, it may give be too good even not yet going to
% 	be converged..., beccause magnetic energy is so small)

%   grep diffe llmf  ---> difference of energies of each iteration.

%   ehf (harris energy)
%   ehk (Hohenberg kohn energy)

% == emax cutoff for APWs. ==
%   We can not use so many APWs in current version,
%   because of overcompleteness (this is because null vector within MTs), 
%   In anyway, use pwemax=3 as standard (test it with 4 or 5).
%   To avoid failure of calculation, we may use smaller MT radius for
%   alkali, and alkali-earth elements. 
%   In feature, I think we can introduce pseudopotentials for these atoms only.

% == Check Used MTO 
%  Near beginig of console output, what MTO you use is shown as: (GaAs case).
%  sugcut:  make orbital-dependent reciprocal vector cutoffs for tol= 1.00E-06
%  spec      l    rsm    eh     gmax    last term    cutoff
%   Ga       0*   1.13  -1.00   6.579    1.19E-06    1459
%   Ga       1*   1.13  -1.00   7.028    1.26E-06    1807
%   Ga       2*   1.13  -1.00   7.475    1.09E-06    2109
%   Ga       3    1.13  -1.00   7.920    1.06E-06    2637
%   Ga       0*   1.13  -2.00   6.579    1.19E-06    1459
%   Ga       1*   1.13  -2.00   7.028    1.26E-06    1807
%   Ga       2    1.13  -2.00   7.475    1.09E-06    2109
%   As       0*   1.18  -1.00   6.300    2.13E-06    1243
%   As       1*   1.18  -1.00   6.720    1.26E-06    1471
%   As       2*   1.18  -1.00   7.140    1.37E-06    1837
%   As       3    1.18  -1.00   7.558    1.05E-06    2229
%   As       0*   1.18  -2.00   6.300    2.13E-06    1243
%   As       1*   1.18  -2.00   6.720    1.26E-06    1471
%   As       2    1.18  -2.00   7.140    1.37E-06    1837


% == gwsc cause error stop.
%  Have you ever changed MTO setting? Consistent with GWinput?

% == QSGW for Fe.
%   It is better to use 3p as core. Furthermore, 3d+4d as valence is better. 
%   Thus we need to set PZ=0,3.9,4.5
%   I also got aware that emax_sigm should be large enough (4$\sim$5 Ry) 
%   to have smooth band dispersion. n1n2n3 can be 10x10x10.

% == RSRNGE: enlarge RSRNGE ===
%   Use RSRNGE=10 or so (in cases, RARNGE=20 or more is required), 
%   for large number of k points. Try and enlarge it if it fails with a
%   message "Exit -1 rdsigm: Bloch sum deviates more than allowed tolerance (tol=5e-6)".
%   We will have to make it automatic in future.
%   Detailed memo (for deverlopers) is at the bottoem of ecalj/Document/BACKUP/MarksOriginalDoc/gw.html.

% == Q0P check
%    In cases, it is better to use Q0Pchoice=2 instead of default Q0Pchoice=1.
%    (For slabs, Q0Pchoice=2 may be better; need check more. In anyway,
%     it is problematic to use unbalanced k points for anisotropic cell).
%     See Copmuter Physics Comm. 176(2007)1-13).

% === When calculation in LDA level fails ===
% when calculation fails in LDA level.
%   (1) smaller MT
%   (2) fewer PW. smaller pwemax.
%   (3) core as semicore.


% ====
% If not stable convergence in gwsc, try to set
% mixbeta 0.5
% (and/or mixpriorit 3 or something)
% at the begining of sigma.


% =======
% cleargw (directory):
% This command clean up up intermediate files under (directory).
% This recursively into deeper level. Be careful, or edit it.
% I use it as '>cleargw .'



% ------------------------------------
% Magnetic moment within MTs are shown as
% ------------------------------------
%  charges:       old           new     
%  smooth      17.240314     17.240740   ...
%  mmom         0.000024     -0.000010   
%  site    1    6.207135      6.206590  
%  mmom         1.062276      1.062991  <--- here
%  site    2    6.207115      6.206834  
%  mmom        -1.062323     -1.062958  <--- here
%  site    3    1.172718      1.172918  
%  mmom         0.000011     -0.000011  
%  site    4    1.172718      1.172918  
%  mmom         0.000011     -0.000011  
% In this case, MTsite1 has 1.062991 and MTsite2 has -1.062958.
% >grep 'lin mix' -A30 llmf 
% can take out this message (if console output is in llmf).



% -----------------------------
% ORBITAL MOMENT in pertubation:
% -----------------------------
% Try 
% >lmf nio --rs=1,0 -vso=1 --quit=band >llmf
% After converged, try
% >grep IORBTM -A20 llmf
% Then llmf shows shows orbital moment in first order perturbation.
% (Here --rs=1,0 read rst.* file but not change it. See >lmf --help.
% --quit=band means quit just after band calculation.)

% == EPS mode,
%   Check Im part of chi0 is smoothly damping at high energy (typically
%   1Ry or larger enengy range). If there is some large Im part remains,
%   something strange (usually due to orthogonality problem of
%   eigenfunctions when you set low q).

%    Related source codes are in ecalj/lm7K/ .
%    A command ecalj/lm7K/ctrlgenM1.py can generate 'standard input file (ctrl file)' 
%    just from a given crystal structure file called as ctrls file. 
%    Binaries are lmf and lmf-MPIK (MPI k-parelell verion).



% \subsection{lmf --help}
% lmf --help show option of --rs=(five numbers); this let lmf know 
% how to read atm.* file which is the initial atom file by lmfa.


\section{Wannier function }
We can generate Wannier functions 
(maximally localized Wannier Functions or similar) 
by a script \exe{genMLWF}. It automatically perform cRPA calculation
sucessively. (If it is not necessary, insert 'exit' in \exe{genMLWF}, after it
performs \verb#lmaxloc2#).

Try to run examples in \verb#ecalj/MATERIALS/Sample_MLWF/#.
Read \verb#README# in it.
To run the script \exe{genMLWF}, we need to get \verb#GWinput#
by editing \verb#GWinput.tmp#. (\verb#mkGWin_lmf2# contains default Wannier section).
%For initial condition, we need
%\begin{verbatim}
%<MLWF>
%5          # gaussian, nwf
%1 1 1 1 1  # nphi(1:nwf)
%9 14  2.0 1.0 phi,phidot and lamda(angs) of gaussian 1, %#iphi(j,i),iphidot(j,i),r0g(j,i),wphi(j,i)
%10 15 2.0 1.0 phi,phidot and lamda(angs) of gaussian 2
%11 16 2.0 1.0 phi,phidot and lamda(angs) of gaussian 3
%12 17 2.0 1.0 phi,phidot and lamda(angs) of gaussian 3
%13 18 2.0 1.0 phi,phidot and lamda(angs) of gaussian 3
%</MLWF>
%\end{verbatim}
In addition, we have some settings (energy windows and so on).
This is the example of the initial conditions for Cu case. 
5 is the number of Wannier function. The most left one means $\phi$ index and the right one of it is $\dot\phi$ index. They are written in the {\bf @MNLA\_CPHI} file.

Then we can run \exe{genMLWF}. 
After it finished, we can analyze it results.
(if you don't need Wannier funciton plot, 
You can skip a line of wanplot in genMLWF. Then we  don't need to set
\verb+vis_wan_*+ options.)

\subsection{lwmatK1 and lwmatK2}
If you input the following command
\begin{verbatim}
>grep Wan lwmatK*
\end{verbatim}

You will get the following results. (This case : Cu cases)
\begin{verbatim}
lwmatK1:  Wannier    1    1   24.644475    0.000000 eV
lwmatK1:  Wannier    1    2   24.644576    0.000000 eV
lwmatK1:  Wannier    1    3   25.471361    0.000000 eV
lwmatK1:  Wannier    1    4   24.644575    0.000000 eV
lwmatK1:  Wannier    1    5   25.470946    0.000000 eV
lwmatK2:  Wannier    1    1    0.000000 eV   -21.263759   -0.000000 eV
lwmatK2:  Wannier    1    2    0.000000 eV   -21.263839    0.000000 eV
lwmatK2:  Wannier    1    3    0.000000 eV   -21.931033   -0.000000 eV
lwmatK2:  Wannier    1    4    0.000000 eV   -21.263839   -0.000000 eV
lwmatK2:  Wannier    1    5    0.000000 eV   -21.930702   -0.000000 eV
\end{verbatim}


\begin{verbatim}

### Wanneir Branch now under developing (imported from T.Miyake's Wannier and H.Kino's).
   A. make at ecalj/fpgw/Wannier/ directory, and do make, and make install. 
      (need to check Makefile first). You first have to install fpgw/exec/ in advance.
   B. Samples are at these directories. 
      MATERIALS/CuMLWFs (small samples),
      MATERIALS/CuMLWF/
      MATERIALS/CuMLWFs/
      MATERIALS/FeMLWF/      
      MATERIALS/NiOMLWF/
      MATERIALS/SrVO3MLWF/
   C. With GWinput and ctrl.*, run 
      >genMLWF
      at these directories.
      In GWinput, we supply settings to generate Wannier funcitons. (Sorry,not documentet yet..)
   D. After genMLWF, do
      >grep Wan lwmatK*
      then compare these with Result.grepWanlwmatK
      These are onsite effective interactions (diagonal part only shown).
      *.xsf are for plotting the Maximally localized Wannier funcitons.
Anyway, documentaion on Wannier is on the way half.
Time consuming part (and also the advantage) is for effective interaction in RPA.
Look into the shell script genMLWF; you can skip last part if you don't need the effective interaction.

\end{verbatim}




\section{How to set local orbitals}
\label{pzsetting}
\begin{verbatim}
  As we stated, do "lmfa |grep conf" to check used MTO basis. 

  We have to set SPEC_ATOM_PZ=?,?,? 
  (they ordered as PZ=s,p,d,f,... ) to set local orbitals.
  
  lmv7 (originally due to ASA in Stuttgart) uses a special terminology
  "continious principle quantum number for each l", which is just
  relatated to the logalismic derivative of radial funcitons at MT
  boundary. It is defined as
   P= principleQuantumNumber + 0.5-1/pi*atan(r* 1/phi dphi/dr),
  where phi is the radial function for each l.   For example, 
   P= n.5 for l=0 of free electron (flat potential) because phi=r^0,
   P= n.25 for l=1 because phi=r^1; 
   P= n.147584 for l=2 because phi=r^2; P=, n.102416 n.077979 for l=3,4.
  (Integer part can be changed). See Logarithmic Derivative Parameters in
  http://titus.phy.qub.ac.uk/packages/LMTO/lmto.html#section2

  Its fractioanl part 0.5-atan(1/phi dphi/dr) is closer to unity for
  core like orbital, but closer to zero for extended orbitals.

  Examples of choice:
  Ga p: in this case, choice 0 or choice 2 is recommended.
      We usually use lo for semi-core, or virtually unoccupied level.

     (0)no lo (4p as valence is default treatment without lo.)
        3p core, 4p valence, no lo: default.
        Then we have choice that lo is set to be for 3p,4p,5p.
     (1)3p lo ---> 4p val (when 3p is treated as valence)
        3d semi core, 4d valence  
        Set PZ=0,3.9 
        (P is not requied to set. *.9 for core like state. It is just an initial condition.)
     (2)5p lo ---> 4p val (PZ>P)
        Set PZ=0,5.5 
        5.5 is just simply given by a guess (no method have yet
		implemented for 
        If 5.2 or something, it may fails
        because of poorness in linear-dependency. We may need to observe
        results should not change so much on the value of PZ.

     (3xxx)4p lo ---> 5p val (we don't use this usually. this is for test purpose)
        4p lo, 5p valence 
        Set PZ=0,4.5 P=0,5.5 (In this case, set P= simultaneously).
        (NOTE: zero for s channel is to use defalut numbers for s)

  Ga d: (in this case, choice 0 or choice 1 is recommended).
     (0)no lo (3d core, 4d valecne, no lo: default.)
          Then we have choice that lo is set to be for 3d,4d,5d.
     (1) 3d lo ---> 4d val  (when 3d is treated as valence)
         Set PZ=0,0,3.9  (P is not required to set)
     (2) 5d lo ---> 4d val  (PZ>P)
         Set PZ=0,0,5.5
     (3xxx) 4d lo ---> 5d val  (this is for test purpose)
         Set PZ=0,0,5.5 P=0,0,4.5
         (NOTE: zero  for s,p are to use defalut numbers )

  %  If you like to read from atm.ga file instead of rst file(if exist).
  %  You have to do lmf --rs=1,1,0,0,1, for example. See lmf --help
  %  Becase rst file keeps the setting of MTO, thus change in ctrl is not
  %  reflected without the above option to lmf.
=============================================================
\end{verbatim}









\newpage
\section{Linear response calculations}
\label{linearr}
With these scripts for linear response calculations, \verb#eps*#,
we can calculate
$\bfq$-dependent dielectric funciton
$\epsilon(\omega ,\bfq)$ (and $v$, $W$) (and $\chi$ for spin fluctuation).
But (because of numerical reason), we can not use $\bfq = 0$ limit.
(if $|\bfq|$ is too small, we have numerical problem, zero divided by
zero, because we have not implemented the version to use $\bfq=0$.)
\begin{itemize}
\item \raw{eps\_lmfh}\\
   Dielectric function epsilon with local field correction.
   Expensive calculation (we may need to reduce number of wing parts in
      future...).

\item \raw{epsPP\_lmfh}\\
   epsilon without local field correction.
   $1- \langle \eiqr |v|\eiqr\rangle  \langle \eiqr| \left( \chi^{0} \right) |\eiqr\rangle$


\item \raw{epsPP\_lmfh\_chipm}

    For spin susceptibility. This essentially calculate non-interacting spin susceptibility.
    Then it is used for the calculation of full spin susceptibility with \verb#util/calj_*.F# programs
    (small quick programs). See spin wave paper.
    See spin susceptibility section Sec.\ref{xxx}.

\item (not maintained now; we will recover this)\raw{eps\_lmfh\_chipm}

    This gives full non-interacting spin susceptibility. Testing.
    We have to determine $U$ (Stoner $I$) for the determination of full spin susceptibility.
    TDLDA? or so?


\item \raw{(This is old mode --- not maintained) epsPP\_lmfh\_chipm\_q}

  For spin susceptibility, 
  spin susceptibility $\langle e^{iqr}| \chi(q,\omega) |e^{iqr} \rangle$
  In this script, You have to assign that isp=1 is majority, isp=2 is minority.
  This is with long wave approximation.  

\end{itemize}

---------------------

\noindent $\bullet$ We use the histogram method (the Hilbert
transformation method); we first calculate its imaginary parts
  with the tetrahedron technique for dielectric functions. 
  Then we get its real part by the Hilbert transformation.\\
  You need to choose \keyw{HisBin\_dw,HisBin\_ratio}. 
  The width of histogram bins are getting larger when omega gets larger.
  dw is the size of histogram-bin width at omega=0. 
  At omega=omg\_c, its width gets twiced.

  To plot dielectric function with reasonable resolution, it might be
  better to set \verb#dw 0.001# and \verb#omg_c 0.1# for example.
  You may have to choose small enough omega 
  for spin wave mode as 0.001 Ry (Or smaller).
  omg\_c is given like 0.05 Ry or so. But sometimes it can be like 1Ry.

  
\noindent $\bullet$ \raw{epsPP\_lmfh} only calculation an matrix element 
  of dielectric funciton for $exp(i \bfq \bfr)$. Thus very faster 
  than \raw{eps\_lmfh} mode. 
  It uses a a special product basis set for cases without inversion
  (problem is in how to expand $\exp(i \bfq \bfr)$ in the MPB;
   the product basis is not from phi and phidot, 
    but from spherical Bessel functions).\\

%\noindent $\bullet$ \keyw{EPSrange, EPSdw} are not used for \verb#*_lmfh_*# scripts.


\begin{verbatim}
In *_lmfh_* modes( I now use little for *_lmf_* modes), you can use small enough delta.
Use small enough delta (=-1e-8 a.u.) for spin wave modes (also you can use it for 
dielectric function and GW).  This is necessary because pole is too smeared 
if you use larger delta.
\end{verbatim}



%%%%%%%%%%%%%%%%%%%%%%%%%%%%%%%%%%%%%%%%%%%%%%%%%%%%%%%%%%%%%%%%
\subsection{eps\_lmfh, epsPP\_lmfh: the dielectric functions}

You can invoke the script, e.g. as "\exe{eps\_lmfh} \ si".

----------------

Specify ${\bf q}$ point in \verb#<QforEPS># or so.
Mesh for $\omega$ is specified by \keyw{dw, omg\_c}.

The obtained data are in {\sf EPS*.dat} and {\sf EPS*.nlfc.dat}.
{\sf EPS*.nlfc.dat} contains the result without local-field correction
{\sf EPS*.dat} contains the result with local-field correction
(this is generated only for \verb#eps_lmfh#. Both of them contains

{\bf q}(1:3), $\omega$, Re($\epsilon$) Im($\epsilon$), Re(1/$\epsilon$), In(1/$\epsilon$)\\
in each line.

%\begin{screen}
% \begin{itemize}
%  \item 
% This code works OK only for ${\bf q}$ is near 0.
% Be careful for ${\bf q} \to 0$ limit. Too small ${\bf q}$ can give strange
% spectrum at high energy (real part is affected by it)\\

% Because ${\bf q} \to 0$ gives too large cancellation effects
% (the denominator and numerator go to zero---it means we need very accurate
% orthogonalization between occupied and unoccupied states).
% This is a kind of disadvantage of our method (though there is an advantage---
% our code can calculate dielectric function even for metal 
% as far as you use large enough number of ${\bf k}$ point.)

% \item
% The calculate of dielectric functions usually requires so many $k$ point. 
% For example, for Si,  \verb#n1 n2 n3 = 4 4 4# is too small. 
% It gives too large dielectric constants $\sim19.4$ though
% the converged value should be $\sim13$. (we need 10x10x10 or more like 20x20x20
% for some reasonable results).
% For GaAs, we observed that reasonable $\epsilon(\omega)$ requires
% rather large number of ${\bf q}$ points like 15x15x15 or 20x20x20
% for \keyw{n1n2n3}. This is too time-consuming to get result
% (but you can use ``very small product basis''(just sp polarization for this purpose;
% it makes speed up so much). Or, you can calculate "$\epsilon(\omega)$ without LFC". 
% See section for \exe{eps\_PP\_lmfh}.

% \item Core orthogonalization problem (only when core2 is used)\\
% ----\keyw{CoreOrth} is not maintained recently ---
%  \keyw{CoreOrth} gives so serious effect for 
% $\epsilon(\omega)$, if you include some cores as "{\bf core2}"
% in the product basis setting.
% (This means that you includes transitions from "{\bf core2} to
% valence" in the calculation of $\epsilon(\omega)$).

% Then you have to use "\raw{CoreOrth} on". Without it,
% you will have rather large imaginary part at rather high energy
% Such transitions from core to higher valence bands
% is artificial due to the incomplete orthogonality
% between core and the higher bands.
% However, shallower $d$ semi-core might be deformed too much
% by this option. Try to plot \io{Core\_*.chk} files, 
% which contains core radial functions. 
% Anyway, it is better to treat shallow core as valence by ``local orbital''.
% \end{itemize}
%\end{screen}




%%%%%%%%%%%%%%%%%
\subsection{epsPP\_lmfh: the dielectric function(No LFC--- faster)}

You can calculate $\epsilon$ without LFC by
{\bf epsPP\_lmfh}. It is very faster than \exe{eps\_lmfh}.

To calculate $\epsilon({\bf q},\omega)$ without LFC accurately,
the best basis set for the expansion of the Coulomb matrix within MT
is apparently not the product basis, but the Bessel functions
corresponding to the plane waves $\exp(i{\bf q r})$.
We use such a basis in this mode. 
However, our experience shows that the changes are little even 
with the usual product basis (we don't describe this here).
%You can test it by the script {\bf epsPtestNoLfc\_nfp}. 
%Please check the script whether it sets the {\bf q}
%point which you want to calculate.
%{\bf epsPtestNoLfc\_nfp} runs \verb#echo 3|hsfp0#. 
%The mode 3 only gives files {\sf  EPSxx.nolfc.dat}.
%
%Apparently you don't needs to do {\bf eps\_lmf} if you just change the file
%{\sf EPS\_cond}; then you just need to run \verb#echo 4|hsfp0#.



%%%%%%%%%%%%%%%%%%%%%%%%%%%%%%%%%%%%%%%%%%%%%%%%%%
\subsection{How to calculate correct dielectric funciton?}

(this subsection is essentially OK... but need to clean it up. dec2014)

\begin{verbatim}

There are prolems to calculate correct epsilon.
At first, we talk about epsPP_lmfh, which is No LFC. Main problem are 

-----------------
1.Convergence for number of k point(specified by n1n2n3). 
  Roughly speaking, 20x20x20 is required for not-so-bad results for Fe and Ni.
  It is better to do 30x30x30 to see convergence check.
  However, in the case of ZB-MnAs (maybe because of simple structure around Ef),
  it requires less q points.

  figs are for GaAs.
  fig001: n1n2n3 convergence for Chi_RegQbz = on  case.
  fig002: n1n2n3 convergence for Chi_RegQbz = off case.
  (Chi_RegQbz in explained in General section in this manual).

  As you see, k points convergence looks a little better in Chi_RegQbz=off
  (mesh not including gamma). However a little ploblem is that its thereshold around 
  0.5eV is too high and slowly changing.

  fig003: Alouanis'(from Arnaud)  vs. ``Chi_RegQbz = on'' vs. ``Chi_RegQbz = off''
  As you see, the threshold of the Red line (20x20x20 Chi_RegQbz=on) and Alouani's 
  are almost the same, but the red line is too oscilating at the low energy part.
  On the other hand, ``Chi_RegQbz = off'' in Green broken line is not so satisfactory
  at the low energy part. 

  fig.gas_eps_kconf.pdf shows the convergence behevior of epsilon for 
  
   
2.$q \to 0$ convergence (this is related to whether Chi_RegQbz=on or off).
  If you use very small q like q=0.001 is GaAs, it can cause a problem.
  Use q=0.01 or larger (maybe q=0.02 or more is safer). 
  Very small q can give numerical error for high-energy region.

  In fig004, we show the high energy tail part of Im $\epsilon(\omega)$ for GaAs case.
  At q=0.01 (this means q= 2*pi/alat * (0 0 0.01)), the imaginary part
  is a little too large . Less than 80eV, q=0.02 gives good results when compared with
  other high q results, though it still has noise above 80eV.
  In fig005, I showed the same results compared with Alouani's (his is up to 40eV).
  Both gives rather good agreements. As you see, q=0.06 or above might be necessary
  to get reasonable convergence for high energy part abouve 40eV.

  We have to be careful for this poorness in high energy part--- it may effect
  low-energy Re[$\epsion$] through KK relation. However this can be very small
  ehough.
  In fig.gas_eps_qconv.jpg, we checked the convergence of eps (\omega=0,q) for q \to 0.
  As you see, it gives convergence, however, q=0.01 is a little out of 
  curve---this should be because of the poorness in the high energy part.
  so q=0.02 or q=0.03 is safer, and you can get eps within 1 percenr accuracy.

3. Including Core for dielectric constant is dangerous. 
   It can cause very poor results if you include core part in GWinput.
   You need to include core just as valence (with local orbital).

   In fig008, we showed core effects. It starts from \approx 16eV 
   (this is core to conduction transition).
   fig007 showd the check about the q point dependence---even with large q,
   it would not change.
   These shows that the core excitation can have larger energy range.
   This is in contrast to the valence case 
   (then the most of excitaion is limited to less than 10eV).
   We have to be careful for such high-energy exciation... The LMTO basis might
   be not so good for high energy.

4. basis set.
   Use QpGcut_psi \approx 3.0 a.u. or so (as same as GW calculation).
   In the case of epsPP* mode, 
   QpGcut_cou can be very small--- In our codes now, 
   ngc>=1 should be for all q vector shown in lqg4gw02 (output of echo 2|qg4gw).
   [In principle, it should be only for the q vector for which we calculate epsilon.
    But there is a technical poorness in our code---
    (maybe) a problem here; the plane-wave part of the eigenfunction generated 
    in lmfgw is not correctly passed to lmf2gw when ngc=0].


-- eps_lmfh: including LFC ----------------------------------
To include eps with LFC, do eps_lmfh. 
But lcutmx=2 seems to be good enough to get 0.5 percent error (maybe better than this).
Test it 10x10x10 or so. (I need to repeat if necessary).
Further you can use smaller QpGcut_cou like 2.2 or so, 
with rather smaller product basis (up to p timed d, not including f).

Note: epsPP_lmfh is designed to use good basis to calculate eps 
without LFC. This is usually in agreement with what you obtained by eps_lmfh;
however it can give slight difference when you use small product basis.


---Summary --------------------
So in conclusion, I think a best way to do is

1. set q=0.02 [q=2pi/alat(0 0 0.02)] or so for GaAs case.
   If you want to check, do q=0.03 and q=0.06 also.

   ``Chi_RegQbz = off'' is better for matrials like GaAs with direct gap.

2. You can use small QpGcut_cou but all ngc should be one or more.

3. As for the Product basis setting in epsPP* scripts, only
   lcutmx and tolerance (this can be like 0.001 or so) are relevant.to determine eps(omega=0, q=0).
   E.g. set lcutmx=4 or so.
   
   5. To get eps with LFC, set QpGcut_cut as xxx, and set lcutmx=2 where
   4. Do nk=20 18 16 and take interpolarion 

   (occupied sp) \timex (unoccupied spd) are included.
   But correct EPS*.nolfc.d is rather from epsPP_lmfh script.

\end{verbatim}


\figp{gas_fig001.eps}

fig001

\figp{gas_fig002.eps}

fig002

%\begin{center}
\figp{gas_fig003.eps}

fig003


\figp{gas_fig004.eps}\\
fig004

\figp{gas_fig005.eps}\\
fig005

\figp{gas_fig007.eps}\\
fig007

\figp{gas_fig008.eps}\\
fig008


\end{document}
